\documentclass[11pt]{article}

    \usepackage[breakable]{tcolorbox}
    \usepackage{parskip} % Stop auto-indenting (to mimic markdown behaviour)
    

    % Basic figure setup, for now with no caption control since it's done
    % automatically by Pandoc (which extracts ![](path) syntax from Markdown).
    \usepackage{graphicx}
    % Keep aspect ratio if custom image width or height is specified
    \setkeys{Gin}{keepaspectratio}
    % Maintain compatibility with old templates. Remove in nbconvert 6.0
    \let\Oldincludegraphics\includegraphics
    % Ensure that by default, figures have no caption (until we provide a
    % proper Figure object with a Caption API and a way to capture that
    % in the conversion process - todo).
    \usepackage{caption}
    \DeclareCaptionFormat{nocaption}{}
    \captionsetup{format=nocaption,aboveskip=0pt,belowskip=0pt}

    \usepackage{float}
    \floatplacement{figure}{H} % forces figures to be placed at the correct location
    \usepackage{xcolor} % Allow colors to be defined
    \usepackage{enumerate} % Needed for markdown enumerations to work
    \usepackage{geometry} % Used to adjust the document margins
    \usepackage{amsmath} % Equations
    \usepackage{amssymb} % Equations
    \usepackage{textcomp} % defines textquotesingle
    % Hack from http://tex.stackexchange.com/a/47451/13684:
    \AtBeginDocument{%
        \def\PYZsq{\textquotesingle}% Upright quotes in Pygmentized code
    }
    \usepackage{upquote} % Upright quotes for verbatim code
    \usepackage{eurosym} % defines \euro

    \usepackage{iftex}
    \ifPDFTeX
        \usepackage[T1]{fontenc}
        \IfFileExists{alphabeta.sty}{
              \usepackage{alphabeta}
          }{
              \usepackage[mathletters]{ucs}
              \usepackage[utf8x]{inputenc}
          }
    \else
        \usepackage{fontspec}
        \usepackage{unicode-math}
    \fi

    \usepackage{fancyvrb} % verbatim replacement that allows latex
    \usepackage{grffile} % extends the file name processing of package graphics
                         % to support a larger range
    \makeatletter % fix for old versions of grffile with XeLaTeX
    \@ifpackagelater{grffile}{2019/11/01}
    {
      % Do nothing on new versions
    }
    {
      \def\Gread@@xetex#1{%
        \IfFileExists{"\Gin@base".bb}%
        {\Gread@eps{\Gin@base.bb}}%
        {\Gread@@xetex@aux#1}%
      }
    }
    \makeatother
    \usepackage[Export]{adjustbox} % Used to constrain images to a maximum size
    \adjustboxset{max size={0.9\linewidth}{0.9\paperheight}}

    % The hyperref package gives us a pdf with properly built
    % internal navigation ('pdf bookmarks' for the table of contents,
    % internal cross-reference links, web links for URLs, etc.)
    \usepackage{hyperref}
    % The default LaTeX title has an obnoxious amount of whitespace. By default,
    % titling removes some of it. It also provides customization options.
    \usepackage{titling}
    \usepackage{longtable} % longtable support required by pandoc >1.10
    \usepackage{booktabs}  % table support for pandoc > 1.12.2
    \usepackage{array}     % table support for pandoc >= 2.11.3
    \usepackage{calc}      % table minipage width calculation for pandoc >= 2.11.1
    \usepackage[inline]{enumitem} % IRkernel/repr support (it uses the enumerate* environment)
    \usepackage[normalem]{ulem} % ulem is needed to support strikethroughs (\sout)
                                % normalem makes italics be italics, not underlines
    \usepackage{soul}      % strikethrough (\st) support for pandoc >= 3.0.0
    \usepackage{mathrsfs}
    

    
    % Colors for the hyperref package
    \definecolor{urlcolor}{rgb}{0,.145,.698}
    \definecolor{linkcolor}{rgb}{.71,0.21,0.01}
    \definecolor{citecolor}{rgb}{.12,.54,.11}

    % ANSI colors
    \definecolor{ansi-black}{HTML}{3E424D}
    \definecolor{ansi-black-intense}{HTML}{282C36}
    \definecolor{ansi-red}{HTML}{E75C58}
    \definecolor{ansi-red-intense}{HTML}{B22B31}
    \definecolor{ansi-green}{HTML}{00A250}
    \definecolor{ansi-green-intense}{HTML}{007427}
    \definecolor{ansi-yellow}{HTML}{DDB62B}
    \definecolor{ansi-yellow-intense}{HTML}{B27D12}
    \definecolor{ansi-blue}{HTML}{208FFB}
    \definecolor{ansi-blue-intense}{HTML}{0065CA}
    \definecolor{ansi-magenta}{HTML}{D160C4}
    \definecolor{ansi-magenta-intense}{HTML}{A03196}
    \definecolor{ansi-cyan}{HTML}{60C6C8}
    \definecolor{ansi-cyan-intense}{HTML}{258F8F}
    \definecolor{ansi-white}{HTML}{C5C1B4}
    \definecolor{ansi-white-intense}{HTML}{A1A6B2}
    \definecolor{ansi-default-inverse-fg}{HTML}{FFFFFF}
    \definecolor{ansi-default-inverse-bg}{HTML}{000000}

    % common color for the border for error outputs.
    \definecolor{outerrorbackground}{HTML}{FFDFDF}

    % commands and environments needed by pandoc snippets
    % extracted from the output of `pandoc -s`
    \providecommand{\tightlist}{%
      \setlength{\itemsep}{0pt}\setlength{\parskip}{0pt}}
    \DefineVerbatimEnvironment{Highlighting}{Verbatim}{commandchars=\\\{\}}
    % Add ',fontsize=\small' for more characters per line
    \newenvironment{Shaded}{}{}
    \newcommand{\KeywordTok}[1]{\textcolor[rgb]{0.00,0.44,0.13}{\textbf{{#1}}}}
    \newcommand{\DataTypeTok}[1]{\textcolor[rgb]{0.56,0.13,0.00}{{#1}}}
    \newcommand{\DecValTok}[1]{\textcolor[rgb]{0.25,0.63,0.44}{{#1}}}
    \newcommand{\BaseNTok}[1]{\textcolor[rgb]{0.25,0.63,0.44}{{#1}}}
    \newcommand{\FloatTok}[1]{\textcolor[rgb]{0.25,0.63,0.44}{{#1}}}
    \newcommand{\CharTok}[1]{\textcolor[rgb]{0.25,0.44,0.63}{{#1}}}
    \newcommand{\StringTok}[1]{\textcolor[rgb]{0.25,0.44,0.63}{{#1}}}
    \newcommand{\CommentTok}[1]{\textcolor[rgb]{0.38,0.63,0.69}{\textit{{#1}}}}
    \newcommand{\OtherTok}[1]{\textcolor[rgb]{0.00,0.44,0.13}{{#1}}}
    \newcommand{\AlertTok}[1]{\textcolor[rgb]{1.00,0.00,0.00}{\textbf{{#1}}}}
    \newcommand{\FunctionTok}[1]{\textcolor[rgb]{0.02,0.16,0.49}{{#1}}}
    \newcommand{\RegionMarkerTok}[1]{{#1}}
    \newcommand{\ErrorTok}[1]{\textcolor[rgb]{1.00,0.00,0.00}{\textbf{{#1}}}}
    \newcommand{\NormalTok}[1]{{#1}}

    % Additional commands for more recent versions of Pandoc
    \newcommand{\ConstantTok}[1]{\textcolor[rgb]{0.53,0.00,0.00}{{#1}}}
    \newcommand{\SpecialCharTok}[1]{\textcolor[rgb]{0.25,0.44,0.63}{{#1}}}
    \newcommand{\VerbatimStringTok}[1]{\textcolor[rgb]{0.25,0.44,0.63}{{#1}}}
    \newcommand{\SpecialStringTok}[1]{\textcolor[rgb]{0.73,0.40,0.53}{{#1}}}
    \newcommand{\ImportTok}[1]{{#1}}
    \newcommand{\DocumentationTok}[1]{\textcolor[rgb]{0.73,0.13,0.13}{\textit{{#1}}}}
    \newcommand{\AnnotationTok}[1]{\textcolor[rgb]{0.38,0.63,0.69}{\textbf{\textit{{#1}}}}}
    \newcommand{\CommentVarTok}[1]{\textcolor[rgb]{0.38,0.63,0.69}{\textbf{\textit{{#1}}}}}
    \newcommand{\VariableTok}[1]{\textcolor[rgb]{0.10,0.09,0.49}{{#1}}}
    \newcommand{\ControlFlowTok}[1]{\textcolor[rgb]{0.00,0.44,0.13}{\textbf{{#1}}}}
    \newcommand{\OperatorTok}[1]{\textcolor[rgb]{0.40,0.40,0.40}{{#1}}}
    \newcommand{\BuiltInTok}[1]{{#1}}
    \newcommand{\ExtensionTok}[1]{{#1}}
    \newcommand{\PreprocessorTok}[1]{\textcolor[rgb]{0.74,0.48,0.00}{{#1}}}
    \newcommand{\AttributeTok}[1]{\textcolor[rgb]{0.49,0.56,0.16}{{#1}}}
    \newcommand{\InformationTok}[1]{\textcolor[rgb]{0.38,0.63,0.69}{\textbf{\textit{{#1}}}}}
    \newcommand{\WarningTok}[1]{\textcolor[rgb]{0.38,0.63,0.69}{\textbf{\textit{{#1}}}}}
    \makeatletter
    \newsavebox\pandoc@box
    \newcommand*\pandocbounded[1]{%
      \sbox\pandoc@box{#1}%
      % scaling factors for width and height
      \Gscale@div\@tempa\textheight{\dimexpr\ht\pandoc@box+\dp\pandoc@box\relax}%
      \Gscale@div\@tempb\linewidth{\wd\pandoc@box}%
      % select the smaller of both
      \ifdim\@tempb\p@<\@tempa\p@
        \let\@tempa\@tempb
      \fi
      % scaling accordingly (\@tempa < 1)
      \ifdim\@tempa\p@<\p@
        \scalebox{\@tempa}{\usebox\pandoc@box}%
      % scaling not needed, use as it is
      \else
        \usebox{\pandoc@box}%
      \fi
    }
    \makeatother

    % Define a nice break command that doesn't care if a line doesn't already
    % exist.
    \def\br{\hspace*{\fill} \\* }
    % Math Jax compatibility definitions
    \def\gt{>}
    \def\lt{<}
    \let\Oldtex\TeX
    \let\Oldlatex\LaTeX
    \renewcommand{\TeX}{\textrm{\Oldtex}}
    \renewcommand{\LaTeX}{\textrm{\Oldlatex}}
    % Document parameters
    % Document title
    \title{06\_notes\_2}
    
    
    
    
    
    
    
% Pygments definitions
\makeatletter
\def\PY@reset{\let\PY@it=\relax \let\PY@bf=\relax%
    \let\PY@ul=\relax \let\PY@tc=\relax%
    \let\PY@bc=\relax \let\PY@ff=\relax}
\def\PY@tok#1{\csname PY@tok@#1\endcsname}
\def\PY@toks#1+{\ifx\relax#1\empty\else%
    \PY@tok{#1}\expandafter\PY@toks\fi}
\def\PY@do#1{\PY@bc{\PY@tc{\PY@ul{%
    \PY@it{\PY@bf{\PY@ff{#1}}}}}}}
\def\PY#1#2{\PY@reset\PY@toks#1+\relax+\PY@do{#2}}

\@namedef{PY@tok@w}{\def\PY@tc##1{\textcolor[rgb]{0.73,0.73,0.73}{##1}}}
\@namedef{PY@tok@c}{\let\PY@it=\textit\def\PY@tc##1{\textcolor[rgb]{0.24,0.48,0.48}{##1}}}
\@namedef{PY@tok@cp}{\def\PY@tc##1{\textcolor[rgb]{0.61,0.40,0.00}{##1}}}
\@namedef{PY@tok@k}{\let\PY@bf=\textbf\def\PY@tc##1{\textcolor[rgb]{0.00,0.50,0.00}{##1}}}
\@namedef{PY@tok@kp}{\def\PY@tc##1{\textcolor[rgb]{0.00,0.50,0.00}{##1}}}
\@namedef{PY@tok@kt}{\def\PY@tc##1{\textcolor[rgb]{0.69,0.00,0.25}{##1}}}
\@namedef{PY@tok@o}{\def\PY@tc##1{\textcolor[rgb]{0.40,0.40,0.40}{##1}}}
\@namedef{PY@tok@ow}{\let\PY@bf=\textbf\def\PY@tc##1{\textcolor[rgb]{0.67,0.13,1.00}{##1}}}
\@namedef{PY@tok@nb}{\def\PY@tc##1{\textcolor[rgb]{0.00,0.50,0.00}{##1}}}
\@namedef{PY@tok@nf}{\def\PY@tc##1{\textcolor[rgb]{0.00,0.00,1.00}{##1}}}
\@namedef{PY@tok@nc}{\let\PY@bf=\textbf\def\PY@tc##1{\textcolor[rgb]{0.00,0.00,1.00}{##1}}}
\@namedef{PY@tok@nn}{\let\PY@bf=\textbf\def\PY@tc##1{\textcolor[rgb]{0.00,0.00,1.00}{##1}}}
\@namedef{PY@tok@ne}{\let\PY@bf=\textbf\def\PY@tc##1{\textcolor[rgb]{0.80,0.25,0.22}{##1}}}
\@namedef{PY@tok@nv}{\def\PY@tc##1{\textcolor[rgb]{0.10,0.09,0.49}{##1}}}
\@namedef{PY@tok@no}{\def\PY@tc##1{\textcolor[rgb]{0.53,0.00,0.00}{##1}}}
\@namedef{PY@tok@nl}{\def\PY@tc##1{\textcolor[rgb]{0.46,0.46,0.00}{##1}}}
\@namedef{PY@tok@ni}{\let\PY@bf=\textbf\def\PY@tc##1{\textcolor[rgb]{0.44,0.44,0.44}{##1}}}
\@namedef{PY@tok@na}{\def\PY@tc##1{\textcolor[rgb]{0.41,0.47,0.13}{##1}}}
\@namedef{PY@tok@nt}{\let\PY@bf=\textbf\def\PY@tc##1{\textcolor[rgb]{0.00,0.50,0.00}{##1}}}
\@namedef{PY@tok@nd}{\def\PY@tc##1{\textcolor[rgb]{0.67,0.13,1.00}{##1}}}
\@namedef{PY@tok@s}{\def\PY@tc##1{\textcolor[rgb]{0.73,0.13,0.13}{##1}}}
\@namedef{PY@tok@sd}{\let\PY@it=\textit\def\PY@tc##1{\textcolor[rgb]{0.73,0.13,0.13}{##1}}}
\@namedef{PY@tok@si}{\let\PY@bf=\textbf\def\PY@tc##1{\textcolor[rgb]{0.64,0.35,0.47}{##1}}}
\@namedef{PY@tok@se}{\let\PY@bf=\textbf\def\PY@tc##1{\textcolor[rgb]{0.67,0.36,0.12}{##1}}}
\@namedef{PY@tok@sr}{\def\PY@tc##1{\textcolor[rgb]{0.64,0.35,0.47}{##1}}}
\@namedef{PY@tok@ss}{\def\PY@tc##1{\textcolor[rgb]{0.10,0.09,0.49}{##1}}}
\@namedef{PY@tok@sx}{\def\PY@tc##1{\textcolor[rgb]{0.00,0.50,0.00}{##1}}}
\@namedef{PY@tok@m}{\def\PY@tc##1{\textcolor[rgb]{0.40,0.40,0.40}{##1}}}
\@namedef{PY@tok@gh}{\let\PY@bf=\textbf\def\PY@tc##1{\textcolor[rgb]{0.00,0.00,0.50}{##1}}}
\@namedef{PY@tok@gu}{\let\PY@bf=\textbf\def\PY@tc##1{\textcolor[rgb]{0.50,0.00,0.50}{##1}}}
\@namedef{PY@tok@gd}{\def\PY@tc##1{\textcolor[rgb]{0.63,0.00,0.00}{##1}}}
\@namedef{PY@tok@gi}{\def\PY@tc##1{\textcolor[rgb]{0.00,0.52,0.00}{##1}}}
\@namedef{PY@tok@gr}{\def\PY@tc##1{\textcolor[rgb]{0.89,0.00,0.00}{##1}}}
\@namedef{PY@tok@ge}{\let\PY@it=\textit}
\@namedef{PY@tok@gs}{\let\PY@bf=\textbf}
\@namedef{PY@tok@ges}{\let\PY@bf=\textbf\let\PY@it=\textit}
\@namedef{PY@tok@gp}{\let\PY@bf=\textbf\def\PY@tc##1{\textcolor[rgb]{0.00,0.00,0.50}{##1}}}
\@namedef{PY@tok@go}{\def\PY@tc##1{\textcolor[rgb]{0.44,0.44,0.44}{##1}}}
\@namedef{PY@tok@gt}{\def\PY@tc##1{\textcolor[rgb]{0.00,0.27,0.87}{##1}}}
\@namedef{PY@tok@err}{\def\PY@bc##1{{\setlength{\fboxsep}{\string -\fboxrule}\fcolorbox[rgb]{1.00,0.00,0.00}{1,1,1}{\strut ##1}}}}
\@namedef{PY@tok@kc}{\let\PY@bf=\textbf\def\PY@tc##1{\textcolor[rgb]{0.00,0.50,0.00}{##1}}}
\@namedef{PY@tok@kd}{\let\PY@bf=\textbf\def\PY@tc##1{\textcolor[rgb]{0.00,0.50,0.00}{##1}}}
\@namedef{PY@tok@kn}{\let\PY@bf=\textbf\def\PY@tc##1{\textcolor[rgb]{0.00,0.50,0.00}{##1}}}
\@namedef{PY@tok@kr}{\let\PY@bf=\textbf\def\PY@tc##1{\textcolor[rgb]{0.00,0.50,0.00}{##1}}}
\@namedef{PY@tok@bp}{\def\PY@tc##1{\textcolor[rgb]{0.00,0.50,0.00}{##1}}}
\@namedef{PY@tok@fm}{\def\PY@tc##1{\textcolor[rgb]{0.00,0.00,1.00}{##1}}}
\@namedef{PY@tok@vc}{\def\PY@tc##1{\textcolor[rgb]{0.10,0.09,0.49}{##1}}}
\@namedef{PY@tok@vg}{\def\PY@tc##1{\textcolor[rgb]{0.10,0.09,0.49}{##1}}}
\@namedef{PY@tok@vi}{\def\PY@tc##1{\textcolor[rgb]{0.10,0.09,0.49}{##1}}}
\@namedef{PY@tok@vm}{\def\PY@tc##1{\textcolor[rgb]{0.10,0.09,0.49}{##1}}}
\@namedef{PY@tok@sa}{\def\PY@tc##1{\textcolor[rgb]{0.73,0.13,0.13}{##1}}}
\@namedef{PY@tok@sb}{\def\PY@tc##1{\textcolor[rgb]{0.73,0.13,0.13}{##1}}}
\@namedef{PY@tok@sc}{\def\PY@tc##1{\textcolor[rgb]{0.73,0.13,0.13}{##1}}}
\@namedef{PY@tok@dl}{\def\PY@tc##1{\textcolor[rgb]{0.73,0.13,0.13}{##1}}}
\@namedef{PY@tok@s2}{\def\PY@tc##1{\textcolor[rgb]{0.73,0.13,0.13}{##1}}}
\@namedef{PY@tok@sh}{\def\PY@tc##1{\textcolor[rgb]{0.73,0.13,0.13}{##1}}}
\@namedef{PY@tok@s1}{\def\PY@tc##1{\textcolor[rgb]{0.73,0.13,0.13}{##1}}}
\@namedef{PY@tok@mb}{\def\PY@tc##1{\textcolor[rgb]{0.40,0.40,0.40}{##1}}}
\@namedef{PY@tok@mf}{\def\PY@tc##1{\textcolor[rgb]{0.40,0.40,0.40}{##1}}}
\@namedef{PY@tok@mh}{\def\PY@tc##1{\textcolor[rgb]{0.40,0.40,0.40}{##1}}}
\@namedef{PY@tok@mi}{\def\PY@tc##1{\textcolor[rgb]{0.40,0.40,0.40}{##1}}}
\@namedef{PY@tok@il}{\def\PY@tc##1{\textcolor[rgb]{0.40,0.40,0.40}{##1}}}
\@namedef{PY@tok@mo}{\def\PY@tc##1{\textcolor[rgb]{0.40,0.40,0.40}{##1}}}
\@namedef{PY@tok@ch}{\let\PY@it=\textit\def\PY@tc##1{\textcolor[rgb]{0.24,0.48,0.48}{##1}}}
\@namedef{PY@tok@cm}{\let\PY@it=\textit\def\PY@tc##1{\textcolor[rgb]{0.24,0.48,0.48}{##1}}}
\@namedef{PY@tok@cpf}{\let\PY@it=\textit\def\PY@tc##1{\textcolor[rgb]{0.24,0.48,0.48}{##1}}}
\@namedef{PY@tok@c1}{\let\PY@it=\textit\def\PY@tc##1{\textcolor[rgb]{0.24,0.48,0.48}{##1}}}
\@namedef{PY@tok@cs}{\let\PY@it=\textit\def\PY@tc##1{\textcolor[rgb]{0.24,0.48,0.48}{##1}}}

\def\PYZbs{\char`\\}
\def\PYZus{\char`\_}
\def\PYZob{\char`\{}
\def\PYZcb{\char`\}}
\def\PYZca{\char`\^}
\def\PYZam{\char`\&}
\def\PYZlt{\char`\<}
\def\PYZgt{\char`\>}
\def\PYZsh{\char`\#}
\def\PYZpc{\char`\%}
\def\PYZdl{\char`\$}
\def\PYZhy{\char`\-}
\def\PYZsq{\char`\'}
\def\PYZdq{\char`\"}
\def\PYZti{\char`\~}
% for compatibility with earlier versions
\def\PYZat{@}
\def\PYZlb{[}
\def\PYZrb{]}
\makeatother


    % For linebreaks inside Verbatim environment from package fancyvrb.
    \makeatletter
        \newbox\Wrappedcontinuationbox
        \newbox\Wrappedvisiblespacebox
        \newcommand*\Wrappedvisiblespace {\textcolor{red}{\textvisiblespace}}
        \newcommand*\Wrappedcontinuationsymbol {\textcolor{red}{\llap{\tiny$\m@th\hookrightarrow$}}}
        \newcommand*\Wrappedcontinuationindent {3ex }
        \newcommand*\Wrappedafterbreak {\kern\Wrappedcontinuationindent\copy\Wrappedcontinuationbox}
        % Take advantage of the already applied Pygments mark-up to insert
        % potential linebreaks for TeX processing.
        %        {, <, #, %, $, ' and ": go to next line.
        %        _, }, ^, &, >, - and ~: stay at end of broken line.
        % Use of \textquotesingle for straight quote.
        \newcommand*\Wrappedbreaksatspecials {%
            \def\PYGZus{\discretionary{\char`\_}{\Wrappedafterbreak}{\char`\_}}%
            \def\PYGZob{\discretionary{}{\Wrappedafterbreak\char`\{}{\char`\{}}%
            \def\PYGZcb{\discretionary{\char`\}}{\Wrappedafterbreak}{\char`\}}}%
            \def\PYGZca{\discretionary{\char`\^}{\Wrappedafterbreak}{\char`\^}}%
            \def\PYGZam{\discretionary{\char`\&}{\Wrappedafterbreak}{\char`\&}}%
            \def\PYGZlt{\discretionary{}{\Wrappedafterbreak\char`\<}{\char`\<}}%
            \def\PYGZgt{\discretionary{\char`\>}{\Wrappedafterbreak}{\char`\>}}%
            \def\PYGZsh{\discretionary{}{\Wrappedafterbreak\char`\#}{\char`\#}}%
            \def\PYGZpc{\discretionary{}{\Wrappedafterbreak\char`\%}{\char`\%}}%
            \def\PYGZdl{\discretionary{}{\Wrappedafterbreak\char`\$}{\char`\$}}%
            \def\PYGZhy{\discretionary{\char`\-}{\Wrappedafterbreak}{\char`\-}}%
            \def\PYGZsq{\discretionary{}{\Wrappedafterbreak\textquotesingle}{\textquotesingle}}%
            \def\PYGZdq{\discretionary{}{\Wrappedafterbreak\char`\"}{\char`\"}}%
            \def\PYGZti{\discretionary{\char`\~}{\Wrappedafterbreak}{\char`\~}}%
        }
        % Some characters . , ; ? ! / are not pygmentized.
        % This macro makes them "active" and they will insert potential linebreaks
        \newcommand*\Wrappedbreaksatpunct {%
            \lccode`\~`\.\lowercase{\def~}{\discretionary{\hbox{\char`\.}}{\Wrappedafterbreak}{\hbox{\char`\.}}}%
            \lccode`\~`\,\lowercase{\def~}{\discretionary{\hbox{\char`\,}}{\Wrappedafterbreak}{\hbox{\char`\,}}}%
            \lccode`\~`\;\lowercase{\def~}{\discretionary{\hbox{\char`\;}}{\Wrappedafterbreak}{\hbox{\char`\;}}}%
            \lccode`\~`\:\lowercase{\def~}{\discretionary{\hbox{\char`\:}}{\Wrappedafterbreak}{\hbox{\char`\:}}}%
            \lccode`\~`\?\lowercase{\def~}{\discretionary{\hbox{\char`\?}}{\Wrappedafterbreak}{\hbox{\char`\?}}}%
            \lccode`\~`\!\lowercase{\def~}{\discretionary{\hbox{\char`\!}}{\Wrappedafterbreak}{\hbox{\char`\!}}}%
            \lccode`\~`\/\lowercase{\def~}{\discretionary{\hbox{\char`\/}}{\Wrappedafterbreak}{\hbox{\char`\/}}}%
            \catcode`\.\active
            \catcode`\,\active
            \catcode`\;\active
            \catcode`\:\active
            \catcode`\?\active
            \catcode`\!\active
            \catcode`\/\active
            \lccode`\~`\~
        }
    \makeatother

    \let\OriginalVerbatim=\Verbatim
    \makeatletter
    \renewcommand{\Verbatim}[1][1]{%
        %\parskip\z@skip
        \sbox\Wrappedcontinuationbox {\Wrappedcontinuationsymbol}%
        \sbox\Wrappedvisiblespacebox {\FV@SetupFont\Wrappedvisiblespace}%
        \def\FancyVerbFormatLine ##1{\hsize\linewidth
            \vtop{\raggedright\hyphenpenalty\z@\exhyphenpenalty\z@
                \doublehyphendemerits\z@\finalhyphendemerits\z@
                \strut ##1\strut}%
        }%
        % If the linebreak is at a space, the latter will be displayed as visible
        % space at end of first line, and a continuation symbol starts next line.
        % Stretch/shrink are however usually zero for typewriter font.
        \def\FV@Space {%
            \nobreak\hskip\z@ plus\fontdimen3\font minus\fontdimen4\font
            \discretionary{\copy\Wrappedvisiblespacebox}{\Wrappedafterbreak}
            {\kern\fontdimen2\font}%
        }%

        % Allow breaks at special characters using \PYG... macros.
        \Wrappedbreaksatspecials
        % Breaks at punctuation characters . , ; ? ! and / need catcode=\active
        \OriginalVerbatim[#1,codes*=\Wrappedbreaksatpunct]%
    }
    \makeatother

    % Exact colors from NB
    \definecolor{incolor}{HTML}{303F9F}
    \definecolor{outcolor}{HTML}{D84315}
    \definecolor{cellborder}{HTML}{CFCFCF}
    \definecolor{cellbackground}{HTML}{F7F7F7}

    % prompt
    \makeatletter
    \newcommand{\boxspacing}{\kern\kvtcb@left@rule\kern\kvtcb@boxsep}
    \makeatother
    \newcommand{\prompt}[4]{
        {\ttfamily\llap{{\color{#2}[#3]:\hspace{3pt}#4}}\vspace{-\baselineskip}}
    }
    

    
    % Prevent overflowing lines due to hard-to-break entities
    \sloppy
    % Setup hyperref package
    \hypersetup{
      breaklinks=true,  % so long urls are correctly broken across lines
      colorlinks=true,
      urlcolor=urlcolor,
      linkcolor=linkcolor,
      citecolor=citecolor,
      }
    % Slightly bigger margins than the latex defaults
    
    \geometry{verbose,tmargin=1in,bmargin=1in,lmargin=1in,rmargin=1in}
    
    

\begin{document}
    
    \maketitle
    
    

    
    \section{Week 6 - Notes: Linear and Angular
Momentum}\label{week-6---notes-linear-and-angular-momentum}

    We've talked about the central conservation laws of classical mechanics:

\begin{itemize}
\tightlist
\item
  Conservation of energy - in a process, if energy is conserved, the
  total energy of the system is the same before and after the process.
  More strongly, in a closed system, the total energy is constant for
  any process (\(dE_{sys}/dt=0\)).
\item
  Conservation of linear momentum - in a process, if momentum is
  conserved, the total momentum of the system is the same before and
  after the process. More strongly, in a closed system, the total
  \emph{vector} momentum is constant for any process
  (\(d\vec{p}_{sys}/dt=0\)).
\item
  Conservation of angular momentum - in a process, if angular momentum
  is conserved, the total angular momentum of the system is the same
  before and after the process. More strongly, in a closed system, the
  total \emph{vector} angular momentum is constant for any process
  (\(d\vec{L}_{sys}/dt=0\)).
\end{itemize}

We've worked with the conservation of energy a lot because it's a
fundamental concept in physics and it lends itself to a scalar equation
analysis. This can be quite a bit simpler in many cases, but an energy
only view of the world can be limiting.

    \subsection{Linear Momentum}\label{linear-momentum}

As we move into the formal study of linear momentum, we will start with
a reminder of the definition of momentum, and the mathematical form of
the conservation of momentum.

Linear momentum is a vector quantity defined as the product of an
object's mass and its velocity. It is denoted by the symbol \(\vec{p}\)
and is defined as:

\[\vec{p} = m\vec{v}\]

where \(m\) is the mass of the object and \(\vec{v}\) is the velocity of
the object. The SI unit of momentum is kg m/s. As we later came to
understand with
\href{https://en.wikipedia.org/wiki/Special_relativity}{Einstein's
special theory of relativity}, this definition of momentum is the
classical limit of the relativistic momentum:

\[\vec{p} = \gamma m\vec{v}\]

where \(\gamma\) is the Lorentz factor,

\[\gamma = \frac{1}{\sqrt{1 - \frac{v^2}{c^2}}}.\]

As you can calculate, the relativistic momentum reduces to the classical
momentum when the velocity is much less than the speed of light. As
\(v/c \rightarrow 0\), \(\gamma \rightarrow 1\), and the relativistic
momentum reduces to the classical momentum.

    \subsubsection{Linear Momentum and Newton's Second
Law}\label{linear-momentum-and-newtons-second-law}

You have seen in our discussion of Newton's Second Law that the net
force on a system is equal to the mass of the system times the
acceleration of the system. This can be written as:

\[\vec{F}_{net}=m\vec{a}.\]

However, this definition and our thinking here with it is a bit limited.
What about systems of objects that are interacting with each other? What
about deformable systems? What happens if something is shedding mass,
like a rocket or jet?

Newton's definition from
\href{https://en.wikipedia.org/wiki/Philosophi\%C3\%A6_Naturalis_Principia_Mathematica}{the
Principia} is a bit more general. He defines the force in terms of the
rate of change of the body's momentum:

\[\vec{F}_{net}=\frac{d\vec{p}}{dt}.\]

We can extend that definition to a system of objects, where the net
force on the system is equal to the rate of change of the total momentum
of the system:

\[\vec{F}_{net}=\frac{d\vec{p}_{sys}}{dt}.\]

The second step might not be obvious, but by working through a few
examples we can see how this is a more useful and general definition of
force.

    \subsubsection{Forces internal to a system zero
out}\label{forces-internal-to-a-system-zero-out}

Consider an abstract system of \(N\) particles. You might think of them
as point particles but they could be extended objects. They experience
outside forces and internal forces; i.e., we go a tag all the particles
in our system and we can tell which ones are interacting with each
other. We can also tell which ones are interacting with the outside
world. This is a bit silly, but it can help us visualize what we are
arguing below.

The total force on the system is given by the sum of all the masses
times the acceleration of each particle:

\[\vec{F}_{total} = \sum_{i=1}^{N} m_i\vec{a}_i = \sum_{i=1}^{N} \vec{F}_i\]

where the last term is the net force on the \(ith\) particle. For a
given object, \(i\), the net force is the sum of all the forces acting
on it, both internal and external,

\[\vec{F}_{i} = \vec{F}_{i}^{int} + \vec{F}_{i}^{ext}.\]

Here these internal forces are pairwise interactions between the
particle \(i\) and every other particle in the system,

\[\vec{F}_{i}^{int} = \sum_{j\neq i}^{N} \vec{F}_{ij},\]

where the sum is over all particles that are not \(i\) because there's
no force between a particle and itself.

Cool, what happens to the internal force equation when we sum over all
particles in the system?

\paragraph{Concrete Examples}\label{concrete-examples}

We have a generic setup, let's see what happens when we apply this to a
few examples: 2 particles, 3 particles, and then N particles.

\subparagraph{Two Particles}\label{two-particles}

With two particles the sum is easy to write out fully.

\[\vec{F}_{int} = \sum_{i=1}^{2} \vec{F}_{i}^{int}\]

\[\vec{F}_{int} = \sum_{i=1}^2 \sum_{j\neq i}^{2} \vec{F}_{ij}\]

\[\vec{F}_{int} = \vec{F}_{12} + \vec{F}_{21} = 0\]

By Newton's Third Law, the force of particle 1 on particle 2 is equal
and opposite to the force of particle 2 on particle 1. The internal
forces cancel out and the net force on the system is the sum of the
external forces.

\[\vec{F}_{12} = -\vec{F}_{21}\]

So here the internal forces sum to zero.

\[\vec{F}_{int} = 0\]

\subparagraph{Three Particles}\label{three-particles}

We can write this out in a similar way.

\[\vec{F}_{int} = \sum_{i=1}^{3} \vec{F}_{i}^{int}\]

\[\vec{F}_{int} = \sum_{i=1}^3 \sum_{j\neq i}^{3} \vec{F}_{ij}\]

\[\vec{F}_{int} = \vec{F}_{12} + \vec{F}_{13} + \vec{F}_{23} + \vec{F}_{21} + \vec{F}_{31} + \vec{F}_{32}\]

We can group these terms by Newton's Third Law pairs.

\[\vec{F}_{int} = (\vec{F}_{12} + \vec{F}_{21}) + (\vec{F}_{13} + \vec{F}_{31}) + (\vec{F}_{23} + \vec{F}_{32}) = 0\]

Every interaction on body \(i\) has a corresponding equal and opposite
interaction on body \(j\), and the internal forces are again zero.

\[\vec{F}_{int} = 0\]

\subparagraph{N Particles}\label{n-particles}

Clearly, there seems to be a pattern here. Namely that the internal
forces are always zero. We can write out the sum for \(N\) particles in
a way that suggests this is always true.

\[\vec{F}_{int} = \sum_{i=1}^{N} \vec{F}_{i}^{int}\]

\[\vec{F}_{int} = \sum_{i=1}^N \sum_{j\neq i}^{N} \vec{F}_{ij}\]

And where we make a switch in the sum terms, so we can counting the
force from each interaction in each term in the sum to make it clear why
the internal forces sum to zero.

\[\vec{F}_{int} = \sum_{i=1}^N \sum_{j>i}^{N} \left(\underbrace{\vec{F}_{ij} + \vec{F}_{ji}}_{\mathrm{always}\,0}\right) = 0\]

Internal forces will always appear as third law pairs, so the internal
interactions will always sum to zero. This is a very powerful result.

\[\vec{F}_{int} = 0\,\mathrm{,\,always}\]

\textbf{For a given system, only external forces can change the
momentum.}

    \subsubsection{Mathematical Form of Conservation of Linear
Momentum}\label{mathematical-form-of-conservation-of-linear-momentum}

Let's look back at the system momentum,

\[\vec{p}_{sys} = \sum_{i=1}^{N} m_i\vec{v}_i = \sum_{i=1}^{N} \vec{p}_i.\]

If we take the time derivative of the system momentum, and assume we
have point particles, so the masses are not changing,

\[\dfrac{d\vec{p}_{sys}}{dt} = \sum_{i=1}^{N} m_i\dfrac{d\vec{v}_i}{dt} = \sum_{i=1}^{N} m_i\vec{a}_i = \sum_{i=1}^{N} \vec{F}_i\]

The net force on the system is given by,

\[\vec{F}_{net} = \vec{F}_{int} + \vec{F}_{ext}.\]

Should there be no external forces, then,

\[\vec{F}_{net} = \vec{F}_{int} = 0.\]

And thus there is no change momentum of the system,

\[\dfrac{d\vec{p}_{sys}}{dt} = 0.\]

So if the system has no external forces, the total momentum of the
system is conserved. We can propose a discrete extension to this form
above where

\[\dfrac{d\vec{p}_{sys}}{dt} \approx \dfrac{\Delta\vec{p}_{sys}}{\Delta t} = 0.\]

And thus, it's easy to see:

\[\Delta \vec{p}_{sys} = \vec{p}_{sys,f} - \vec{p}_{sys,i} = 0.\]

If there are external forces, then we also have a prediction equation
for how the energy will change in a small time step \(\Delta t\):

\[\Delta \vec{p}_{sys} = \vec{p}_{sys,f} - \vec{p}_{sys,i} = \sum_{i=1}^{N} \vec{F}_{ext,i}\Delta t,\]

so that,

\[\vec{p}_{sys,f} = \vec{p}_{sys,i} + \vec{F}_{ext}\Delta t.\]

    \subsection{Angular Momentum}\label{angular-momentum}

\href{https://en.wikipedia.org/wiki/Angular_momentum}{Angular momentum}
is a complex and rich quantity that has deep connections to the shape
and structure of a system. The ``configuration'' or how it is
distributed in space can have a big impact on the dynamics of a system.
Our study of classical angular momentum will be a stepping stone to our
study of
\href{https://en.wikipedia.org/wiki/Angular_momentum_operator}{quantum
angular momentum} and the
\href{\%5Bhttps://en.wikipedia.org/wiki/Spin_(physics)\%5D}{spin} of
particles.

:::\{admonition\} Quantum Mechanical Spin :class: information
\href{https://en.wikipedia.org/wiki/Spin_(physics)}{Spin} is a quantum
mechanical property that is not related to the rotation of a particle,
but it is a form of angular momentum, and it's essential to the
structure of the universe - it tells us if we have
\href{https://en.wikipedia.org/wiki/Fermion}{fermionic} or
\href{https://en.wikipedia.org/wiki/Boson}{bosonic} particles, it is
what gives us the
\href{https://en.wikipedia.org/wiki/Pauli_exclusion_principle}{Pauli
Exclusion Principle}, and it is what gives us the
\href{https://en.wikipedia.org/wiki/Zeeman_effect}{Zeeman Effect} and
the \href{https://en.wikipedia.org/wiki/Stark_effect}{Stark Effect}. :::

    For the moment, we will limit ourselves to classical angular momentum
and we will focus on the abstract case of a single particle. As we work
through the semester, we will revisit angular momentum and introduce how
to work with distributions of mass and extended objects.

    \subsubsection{Definition of Angular
Momentum}\label{definition-of-angular-momentum}

For a particle with a momentum \(\vec{p}\), the angular momentum is
defined as the cross product of the position vector \(\vec{r}\) and the
momentum vector \(\vec{p}\),

\[\vec{L} = \vec{r} \times \vec{p} = m \left(\vec{r} \times \vec{v}\right).\]

This is a quantity that depends on the location of the particle relative
origin of coordinates. This means you have some latitude in choosing the
origin of coordinates, and you can choose the origin to simplify the
problem.

This also means the angular momentum is a vector quantity, and it points
in the direction perpendicular to the plane defined by the position and
momentum vectors.

    \subsubsection{When is Angular Momentum
Conserved?}\label{when-is-angular-momentum-conserved}

We can ask this by computing the time derivative of the angular
momentum,

\[\dfrac{d\vec{L}}{dt} = \dfrac{d}{dt}\left(\vec{r} \times \vec{p}\right) = 0?\]

We did a calculation like this on a homework where we computed

\[\dfrac{d}{dt}\left(\vec{a}\times\vec{b}\right) = \vec{a}\times\dfrac{d\vec{b}}{dt} + \dfrac{d\vec{a}}{dt}\times\vec{b}.\]

Let's apply it here:

\[\dfrac{d\vec{L}}{dt} = \dfrac{d}{dt}\left(\vec{r} \times \vec{p}\right) = \dfrac{d\vec{r}}{dt} \times \vec{p} + \vec{r} \times \dfrac{d\vec{p}}{dt}.\]

If we assume that \(\dot{m}=0\), then we can write the time derivative
of the momentum as,

\[\dfrac{d\vec{p}}{dt} = \dfrac{d}{dt}\left(m\vec{v}\right) = m\dfrac{d\vec{v}}{dt}\]

We group the terms in the time derivative of the angular momentum,

\[\dfrac{d\vec{L}}{dt} = m \underbrace{\dfrac{d\vec{r}}{dt} \times \vec{v}}_{=0} + m\vec{r} \times \dfrac{d\vec{v}}{dt}.\]

The first term the cross product of the velocity with itself
\(\vec{v}\times\vec{v}\), and is zero, and the second term is the cross
product of the position vector with the acceleration,

\[\dfrac{d\vec{L}}{dt} = m\vec{r} \times \dfrac{d\vec{v}}{dt} = \vec{r}\times m\vec{a}.\]

So the time derivative of the angular momentum is the net torque on the
system!

\[\vec{\tau}_{net} = \vec{r} \times \vec{F}_{net}.\]

\[\dfrac{d\vec{L}}{dt} = \vec{r}\times m\vec{a} = \vec{r} \times \vec{F}_{net}\]

\[\dfrac{d\vec{L}}{dt} = \vec{\tau}_{net}.\]

If the net torque on the system is zero, then the angular momentum is
conserved, and it is a constant of the motion.

\[\dfrac{d\vec{L}_{sys}}{dt} = 0.\]

\[\Delta \vec{L}_{sys} = \vec{L}_{sys,f} - \vec{L}_{sys,i} = 0.\]

If there's a net torque, we have a discrete update equation for the
angular momentum,

\[\vec{L}_{sys,f} = \vec{L}_{sys,i} + \vec{\tau}_{net}\Delta t.\]

\subsubsection{Are we sure there are no internal torques that
matter?}\label{are-we-sure-there-are-no-internal-torques-that-matter}

We can ask the same question we asked about internal forces. Are there
internal torques that matter? As before, let us define the total force
on particle \(i\) as the sum of internal and external forces,

\[\vec{F}_{i} = \vec{F}_{i}^{int} + \vec{F}_{i}^{ext}.\]

We assume there are no external forces, and so the net force on the
system is the sum of the internal forces,

\[\vec{F}_{net} = \vec{F}_{int}.\]

For given object, we observe an angular momentum \(\vec{l}_i\) that is
the cross product of the position vector and the momentum vector. So the
time derivative of the \(i\)th particle's angular momentum is,

\[\dfrac{d\vec{l}_i}{dt} = \vec{r}_i \times \vec{F}_i.\]

If the total angular momentum of the system is the sum of the angular
momenta of the particles,

\[\vec{L} = \sum_{i=1}^{N} \vec{l}_i,\]

then the time derivative of the total angular momentum is,

\[\dfrac{d\vec{L}}{dt} = \sum_{i=1}^{N} \dfrac{d\vec{l}_i}{dt} = \sum_{i=1}^{N} \vec{r}_i \times \vec{F}_i.\]

Recall that \(\vec{F}_i = \sum_{j\neq i}^{N} \vec{F}_{ij}\). So we can
rewrite the time derivative of the total angular momentum as,

\[\dfrac{d\vec{L}}{dt} = \sum_{i=1}^{N} \sum_{j\neq i}^{N} \vec{r}_i  \times \vec{F}_{ij} = \sum_{i=1}^N \sum_{j >i}^{N} \left(\vec{r}_i \times \vec{F}_{ij} + \vec{r}_j \times \vec{F}_{ji}\right)\]

But note that
\(\vec{r}_i \times \vec{F}_{ij} = -\vec{r}_j \times \vec{F}_{ji}\), so
the resulting expression gives us,

\[\dfrac{d\vec{L}}{dt} =\sum_{i=1}^N \sum_{j>i}^{N} \left(\vec{r}_i - \vec{r}_j\right)\times \vec{F}_{ij}.\]

So if the internal forces are parallel to the separation between the
particles, then the internal torques sum to zero, and the total angular
momentum of the system is conserved. So things like the gravitational
force, the electric force, and spring forces are all internal forces
that do not contribute to the net torque on the system.

And thus,

\[\dfrac{d\vec{L}_{sys}}{dt} = 0.\]

    \begin{tcolorbox}[breakable, size=fbox, boxrule=1pt, pad at break*=1mm,colback=cellbackground, colframe=cellborder]
\prompt{In}{incolor}{ }{\boxspacing}
\begin{Verbatim}[commandchars=\\\{\}]

\end{Verbatim}
\end{tcolorbox}

    \begin{tcolorbox}[breakable, size=fbox, boxrule=1pt, pad at break*=1mm,colback=cellbackground, colframe=cellborder]
\prompt{In}{incolor}{ }{\boxspacing}
\begin{Verbatim}[commandchars=\\\{\}]

\end{Verbatim}
\end{tcolorbox}

    


    % Add a bibliography block to the postdoc
    
    
    
\end{document}
