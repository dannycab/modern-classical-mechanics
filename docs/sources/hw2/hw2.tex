\documentclass[11pt]{article}

    \usepackage[breakable]{tcolorbox}
    \usepackage{parskip} % Stop auto-indenting (to mimic markdown behaviour)
    

    % Basic figure setup, for now with no caption control since it's done
    % automatically by Pandoc (which extracts ![](path) syntax from Markdown).
    \usepackage{graphicx}
    % Keep aspect ratio if custom image width or height is specified
    \setkeys{Gin}{keepaspectratio}
    % Maintain compatibility with old templates. Remove in nbconvert 6.0
    \let\Oldincludegraphics\includegraphics
    % Ensure that by default, figures have no caption (until we provide a
    % proper Figure object with a Caption API and a way to capture that
    % in the conversion process - todo).
    \usepackage{caption}
    \DeclareCaptionFormat{nocaption}{}
    \captionsetup{format=nocaption,aboveskip=0pt,belowskip=0pt}

    \usepackage{float}
    \floatplacement{figure}{H} % forces figures to be placed at the correct location
    \usepackage{xcolor} % Allow colors to be defined
    \usepackage{enumerate} % Needed for markdown enumerations to work
    \usepackage{geometry} % Used to adjust the document margins
    \usepackage{amsmath} % Equations
    \usepackage{amssymb} % Equations
    \usepackage{textcomp} % defines textquotesingle
    % Hack from http://tex.stackexchange.com/a/47451/13684:
    \AtBeginDocument{%
        \def\PYZsq{\textquotesingle}% Upright quotes in Pygmentized code
    }
    \usepackage{upquote} % Upright quotes for verbatim code
    \usepackage{eurosym} % defines \euro

    \usepackage{iftex}
    \ifPDFTeX
        \usepackage[T1]{fontenc}
        \IfFileExists{alphabeta.sty}{
              \usepackage{alphabeta}
          }{
              \usepackage[mathletters]{ucs}
              \usepackage[utf8x]{inputenc}
          }
    \else
        \usepackage{fontspec}
        \usepackage{unicode-math}
    \fi

    \usepackage{fancyvrb} % verbatim replacement that allows latex
    \usepackage{grffile} % extends the file name processing of package graphics
                         % to support a larger range
    \makeatletter % fix for old versions of grffile with XeLaTeX
    \@ifpackagelater{grffile}{2019/11/01}
    {
      % Do nothing on new versions
    }
    {
      \def\Gread@@xetex#1{%
        \IfFileExists{"\Gin@base".bb}%
        {\Gread@eps{\Gin@base.bb}}%
        {\Gread@@xetex@aux#1}%
      }
    }
    \makeatother
    \usepackage[Export]{adjustbox} % Used to constrain images to a maximum size
    \adjustboxset{max size={0.9\linewidth}{0.9\paperheight}}

    % The hyperref package gives us a pdf with properly built
    % internal navigation ('pdf bookmarks' for the table of contents,
    % internal cross-reference links, web links for URLs, etc.)
    \usepackage{hyperref}
    % The default LaTeX title has an obnoxious amount of whitespace. By default,
    % titling removes some of it. It also provides customization options.
    \usepackage{titling}
    \usepackage{longtable} % longtable support required by pandoc >1.10
    \usepackage{booktabs}  % table support for pandoc > 1.12.2
    \usepackage{array}     % table support for pandoc >= 2.11.3
    \usepackage{calc}      % table minipage width calculation for pandoc >= 2.11.1
    \usepackage[inline]{enumitem} % IRkernel/repr support (it uses the enumerate* environment)
    \usepackage[normalem]{ulem} % ulem is needed to support strikethroughs (\sout)
                                % normalem makes italics be italics, not underlines
    \usepackage{soul}      % strikethrough (\st) support for pandoc >= 3.0.0
    \usepackage{mathrsfs}
    

    
    % Colors for the hyperref package
    \definecolor{urlcolor}{rgb}{0,.145,.698}
    \definecolor{linkcolor}{rgb}{.71,0.21,0.01}
    \definecolor{citecolor}{rgb}{.12,.54,.11}

    % ANSI colors
    \definecolor{ansi-black}{HTML}{3E424D}
    \definecolor{ansi-black-intense}{HTML}{282C36}
    \definecolor{ansi-red}{HTML}{E75C58}
    \definecolor{ansi-red-intense}{HTML}{B22B31}
    \definecolor{ansi-green}{HTML}{00A250}
    \definecolor{ansi-green-intense}{HTML}{007427}
    \definecolor{ansi-yellow}{HTML}{DDB62B}
    \definecolor{ansi-yellow-intense}{HTML}{B27D12}
    \definecolor{ansi-blue}{HTML}{208FFB}
    \definecolor{ansi-blue-intense}{HTML}{0065CA}
    \definecolor{ansi-magenta}{HTML}{D160C4}
    \definecolor{ansi-magenta-intense}{HTML}{A03196}
    \definecolor{ansi-cyan}{HTML}{60C6C8}
    \definecolor{ansi-cyan-intense}{HTML}{258F8F}
    \definecolor{ansi-white}{HTML}{C5C1B4}
    \definecolor{ansi-white-intense}{HTML}{A1A6B2}
    \definecolor{ansi-default-inverse-fg}{HTML}{FFFFFF}
    \definecolor{ansi-default-inverse-bg}{HTML}{000000}

    % common color for the border for error outputs.
    \definecolor{outerrorbackground}{HTML}{FFDFDF}

    % commands and environments needed by pandoc snippets
    % extracted from the output of `pandoc -s`
    \providecommand{\tightlist}{%
      \setlength{\itemsep}{0pt}\setlength{\parskip}{0pt}}
    \DefineVerbatimEnvironment{Highlighting}{Verbatim}{commandchars=\\\{\}}
    % Add ',fontsize=\small' for more characters per line
    \newenvironment{Shaded}{}{}
    \newcommand{\KeywordTok}[1]{\textcolor[rgb]{0.00,0.44,0.13}{\textbf{{#1}}}}
    \newcommand{\DataTypeTok}[1]{\textcolor[rgb]{0.56,0.13,0.00}{{#1}}}
    \newcommand{\DecValTok}[1]{\textcolor[rgb]{0.25,0.63,0.44}{{#1}}}
    \newcommand{\BaseNTok}[1]{\textcolor[rgb]{0.25,0.63,0.44}{{#1}}}
    \newcommand{\FloatTok}[1]{\textcolor[rgb]{0.25,0.63,0.44}{{#1}}}
    \newcommand{\CharTok}[1]{\textcolor[rgb]{0.25,0.44,0.63}{{#1}}}
    \newcommand{\StringTok}[1]{\textcolor[rgb]{0.25,0.44,0.63}{{#1}}}
    \newcommand{\CommentTok}[1]{\textcolor[rgb]{0.38,0.63,0.69}{\textit{{#1}}}}
    \newcommand{\OtherTok}[1]{\textcolor[rgb]{0.00,0.44,0.13}{{#1}}}
    \newcommand{\AlertTok}[1]{\textcolor[rgb]{1.00,0.00,0.00}{\textbf{{#1}}}}
    \newcommand{\FunctionTok}[1]{\textcolor[rgb]{0.02,0.16,0.49}{{#1}}}
    \newcommand{\RegionMarkerTok}[1]{{#1}}
    \newcommand{\ErrorTok}[1]{\textcolor[rgb]{1.00,0.00,0.00}{\textbf{{#1}}}}
    \newcommand{\NormalTok}[1]{{#1}}

    % Additional commands for more recent versions of Pandoc
    \newcommand{\ConstantTok}[1]{\textcolor[rgb]{0.53,0.00,0.00}{{#1}}}
    \newcommand{\SpecialCharTok}[1]{\textcolor[rgb]{0.25,0.44,0.63}{{#1}}}
    \newcommand{\VerbatimStringTok}[1]{\textcolor[rgb]{0.25,0.44,0.63}{{#1}}}
    \newcommand{\SpecialStringTok}[1]{\textcolor[rgb]{0.73,0.40,0.53}{{#1}}}
    \newcommand{\ImportTok}[1]{{#1}}
    \newcommand{\DocumentationTok}[1]{\textcolor[rgb]{0.73,0.13,0.13}{\textit{{#1}}}}
    \newcommand{\AnnotationTok}[1]{\textcolor[rgb]{0.38,0.63,0.69}{\textbf{\textit{{#1}}}}}
    \newcommand{\CommentVarTok}[1]{\textcolor[rgb]{0.38,0.63,0.69}{\textbf{\textit{{#1}}}}}
    \newcommand{\VariableTok}[1]{\textcolor[rgb]{0.10,0.09,0.49}{{#1}}}
    \newcommand{\ControlFlowTok}[1]{\textcolor[rgb]{0.00,0.44,0.13}{\textbf{{#1}}}}
    \newcommand{\OperatorTok}[1]{\textcolor[rgb]{0.40,0.40,0.40}{{#1}}}
    \newcommand{\BuiltInTok}[1]{{#1}}
    \newcommand{\ExtensionTok}[1]{{#1}}
    \newcommand{\PreprocessorTok}[1]{\textcolor[rgb]{0.74,0.48,0.00}{{#1}}}
    \newcommand{\AttributeTok}[1]{\textcolor[rgb]{0.49,0.56,0.16}{{#1}}}
    \newcommand{\InformationTok}[1]{\textcolor[rgb]{0.38,0.63,0.69}{\textbf{\textit{{#1}}}}}
    \newcommand{\WarningTok}[1]{\textcolor[rgb]{0.38,0.63,0.69}{\textbf{\textit{{#1}}}}}
    \makeatletter
    \newsavebox\pandoc@box
    \newcommand*\pandocbounded[1]{%
      \sbox\pandoc@box{#1}%
      % scaling factors for width and height
      \Gscale@div\@tempa\textheight{\dimexpr\ht\pandoc@box+\dp\pandoc@box\relax}%
      \Gscale@div\@tempb\linewidth{\wd\pandoc@box}%
      % select the smaller of both
      \ifdim\@tempb\p@<\@tempa\p@
        \let\@tempa\@tempb
      \fi
      % scaling accordingly (\@tempa < 1)
      \ifdim\@tempa\p@<\p@
        \scalebox{\@tempa}{\usebox\pandoc@box}%
      % scaling not needed, use as it is
      \else
        \usebox{\pandoc@box}%
      \fi
    }
    \makeatother

    % Define a nice break command that doesn't care if a line doesn't already
    % exist.
    \def\br{\hspace*{\fill} \\* }
    % Math Jax compatibility definitions
    \def\gt{>}
    \def\lt{<}
    \let\Oldtex\TeX
    \let\Oldlatex\LaTeX
    \renewcommand{\TeX}{\textrm{\Oldtex}}
    \renewcommand{\LaTeX}{\textrm{\Oldlatex}}
    % Document parameters
    % Document title
    \title{hw2}
    
    
    
    
    
    
    
% Pygments definitions
\makeatletter
\def\PY@reset{\let\PY@it=\relax \let\PY@bf=\relax%
    \let\PY@ul=\relax \let\PY@tc=\relax%
    \let\PY@bc=\relax \let\PY@ff=\relax}
\def\PY@tok#1{\csname PY@tok@#1\endcsname}
\def\PY@toks#1+{\ifx\relax#1\empty\else%
    \PY@tok{#1}\expandafter\PY@toks\fi}
\def\PY@do#1{\PY@bc{\PY@tc{\PY@ul{%
    \PY@it{\PY@bf{\PY@ff{#1}}}}}}}
\def\PY#1#2{\PY@reset\PY@toks#1+\relax+\PY@do{#2}}

\@namedef{PY@tok@w}{\def\PY@tc##1{\textcolor[rgb]{0.73,0.73,0.73}{##1}}}
\@namedef{PY@tok@c}{\let\PY@it=\textit\def\PY@tc##1{\textcolor[rgb]{0.24,0.48,0.48}{##1}}}
\@namedef{PY@tok@cp}{\def\PY@tc##1{\textcolor[rgb]{0.61,0.40,0.00}{##1}}}
\@namedef{PY@tok@k}{\let\PY@bf=\textbf\def\PY@tc##1{\textcolor[rgb]{0.00,0.50,0.00}{##1}}}
\@namedef{PY@tok@kp}{\def\PY@tc##1{\textcolor[rgb]{0.00,0.50,0.00}{##1}}}
\@namedef{PY@tok@kt}{\def\PY@tc##1{\textcolor[rgb]{0.69,0.00,0.25}{##1}}}
\@namedef{PY@tok@o}{\def\PY@tc##1{\textcolor[rgb]{0.40,0.40,0.40}{##1}}}
\@namedef{PY@tok@ow}{\let\PY@bf=\textbf\def\PY@tc##1{\textcolor[rgb]{0.67,0.13,1.00}{##1}}}
\@namedef{PY@tok@nb}{\def\PY@tc##1{\textcolor[rgb]{0.00,0.50,0.00}{##1}}}
\@namedef{PY@tok@nf}{\def\PY@tc##1{\textcolor[rgb]{0.00,0.00,1.00}{##1}}}
\@namedef{PY@tok@nc}{\let\PY@bf=\textbf\def\PY@tc##1{\textcolor[rgb]{0.00,0.00,1.00}{##1}}}
\@namedef{PY@tok@nn}{\let\PY@bf=\textbf\def\PY@tc##1{\textcolor[rgb]{0.00,0.00,1.00}{##1}}}
\@namedef{PY@tok@ne}{\let\PY@bf=\textbf\def\PY@tc##1{\textcolor[rgb]{0.80,0.25,0.22}{##1}}}
\@namedef{PY@tok@nv}{\def\PY@tc##1{\textcolor[rgb]{0.10,0.09,0.49}{##1}}}
\@namedef{PY@tok@no}{\def\PY@tc##1{\textcolor[rgb]{0.53,0.00,0.00}{##1}}}
\@namedef{PY@tok@nl}{\def\PY@tc##1{\textcolor[rgb]{0.46,0.46,0.00}{##1}}}
\@namedef{PY@tok@ni}{\let\PY@bf=\textbf\def\PY@tc##1{\textcolor[rgb]{0.44,0.44,0.44}{##1}}}
\@namedef{PY@tok@na}{\def\PY@tc##1{\textcolor[rgb]{0.41,0.47,0.13}{##1}}}
\@namedef{PY@tok@nt}{\let\PY@bf=\textbf\def\PY@tc##1{\textcolor[rgb]{0.00,0.50,0.00}{##1}}}
\@namedef{PY@tok@nd}{\def\PY@tc##1{\textcolor[rgb]{0.67,0.13,1.00}{##1}}}
\@namedef{PY@tok@s}{\def\PY@tc##1{\textcolor[rgb]{0.73,0.13,0.13}{##1}}}
\@namedef{PY@tok@sd}{\let\PY@it=\textit\def\PY@tc##1{\textcolor[rgb]{0.73,0.13,0.13}{##1}}}
\@namedef{PY@tok@si}{\let\PY@bf=\textbf\def\PY@tc##1{\textcolor[rgb]{0.64,0.35,0.47}{##1}}}
\@namedef{PY@tok@se}{\let\PY@bf=\textbf\def\PY@tc##1{\textcolor[rgb]{0.67,0.36,0.12}{##1}}}
\@namedef{PY@tok@sr}{\def\PY@tc##1{\textcolor[rgb]{0.64,0.35,0.47}{##1}}}
\@namedef{PY@tok@ss}{\def\PY@tc##1{\textcolor[rgb]{0.10,0.09,0.49}{##1}}}
\@namedef{PY@tok@sx}{\def\PY@tc##1{\textcolor[rgb]{0.00,0.50,0.00}{##1}}}
\@namedef{PY@tok@m}{\def\PY@tc##1{\textcolor[rgb]{0.40,0.40,0.40}{##1}}}
\@namedef{PY@tok@gh}{\let\PY@bf=\textbf\def\PY@tc##1{\textcolor[rgb]{0.00,0.00,0.50}{##1}}}
\@namedef{PY@tok@gu}{\let\PY@bf=\textbf\def\PY@tc##1{\textcolor[rgb]{0.50,0.00,0.50}{##1}}}
\@namedef{PY@tok@gd}{\def\PY@tc##1{\textcolor[rgb]{0.63,0.00,0.00}{##1}}}
\@namedef{PY@tok@gi}{\def\PY@tc##1{\textcolor[rgb]{0.00,0.52,0.00}{##1}}}
\@namedef{PY@tok@gr}{\def\PY@tc##1{\textcolor[rgb]{0.89,0.00,0.00}{##1}}}
\@namedef{PY@tok@ge}{\let\PY@it=\textit}
\@namedef{PY@tok@gs}{\let\PY@bf=\textbf}
\@namedef{PY@tok@ges}{\let\PY@bf=\textbf\let\PY@it=\textit}
\@namedef{PY@tok@gp}{\let\PY@bf=\textbf\def\PY@tc##1{\textcolor[rgb]{0.00,0.00,0.50}{##1}}}
\@namedef{PY@tok@go}{\def\PY@tc##1{\textcolor[rgb]{0.44,0.44,0.44}{##1}}}
\@namedef{PY@tok@gt}{\def\PY@tc##1{\textcolor[rgb]{0.00,0.27,0.87}{##1}}}
\@namedef{PY@tok@err}{\def\PY@bc##1{{\setlength{\fboxsep}{\string -\fboxrule}\fcolorbox[rgb]{1.00,0.00,0.00}{1,1,1}{\strut ##1}}}}
\@namedef{PY@tok@kc}{\let\PY@bf=\textbf\def\PY@tc##1{\textcolor[rgb]{0.00,0.50,0.00}{##1}}}
\@namedef{PY@tok@kd}{\let\PY@bf=\textbf\def\PY@tc##1{\textcolor[rgb]{0.00,0.50,0.00}{##1}}}
\@namedef{PY@tok@kn}{\let\PY@bf=\textbf\def\PY@tc##1{\textcolor[rgb]{0.00,0.50,0.00}{##1}}}
\@namedef{PY@tok@kr}{\let\PY@bf=\textbf\def\PY@tc##1{\textcolor[rgb]{0.00,0.50,0.00}{##1}}}
\@namedef{PY@tok@bp}{\def\PY@tc##1{\textcolor[rgb]{0.00,0.50,0.00}{##1}}}
\@namedef{PY@tok@fm}{\def\PY@tc##1{\textcolor[rgb]{0.00,0.00,1.00}{##1}}}
\@namedef{PY@tok@vc}{\def\PY@tc##1{\textcolor[rgb]{0.10,0.09,0.49}{##1}}}
\@namedef{PY@tok@vg}{\def\PY@tc##1{\textcolor[rgb]{0.10,0.09,0.49}{##1}}}
\@namedef{PY@tok@vi}{\def\PY@tc##1{\textcolor[rgb]{0.10,0.09,0.49}{##1}}}
\@namedef{PY@tok@vm}{\def\PY@tc##1{\textcolor[rgb]{0.10,0.09,0.49}{##1}}}
\@namedef{PY@tok@sa}{\def\PY@tc##1{\textcolor[rgb]{0.73,0.13,0.13}{##1}}}
\@namedef{PY@tok@sb}{\def\PY@tc##1{\textcolor[rgb]{0.73,0.13,0.13}{##1}}}
\@namedef{PY@tok@sc}{\def\PY@tc##1{\textcolor[rgb]{0.73,0.13,0.13}{##1}}}
\@namedef{PY@tok@dl}{\def\PY@tc##1{\textcolor[rgb]{0.73,0.13,0.13}{##1}}}
\@namedef{PY@tok@s2}{\def\PY@tc##1{\textcolor[rgb]{0.73,0.13,0.13}{##1}}}
\@namedef{PY@tok@sh}{\def\PY@tc##1{\textcolor[rgb]{0.73,0.13,0.13}{##1}}}
\@namedef{PY@tok@s1}{\def\PY@tc##1{\textcolor[rgb]{0.73,0.13,0.13}{##1}}}
\@namedef{PY@tok@mb}{\def\PY@tc##1{\textcolor[rgb]{0.40,0.40,0.40}{##1}}}
\@namedef{PY@tok@mf}{\def\PY@tc##1{\textcolor[rgb]{0.40,0.40,0.40}{##1}}}
\@namedef{PY@tok@mh}{\def\PY@tc##1{\textcolor[rgb]{0.40,0.40,0.40}{##1}}}
\@namedef{PY@tok@mi}{\def\PY@tc##1{\textcolor[rgb]{0.40,0.40,0.40}{##1}}}
\@namedef{PY@tok@il}{\def\PY@tc##1{\textcolor[rgb]{0.40,0.40,0.40}{##1}}}
\@namedef{PY@tok@mo}{\def\PY@tc##1{\textcolor[rgb]{0.40,0.40,0.40}{##1}}}
\@namedef{PY@tok@ch}{\let\PY@it=\textit\def\PY@tc##1{\textcolor[rgb]{0.24,0.48,0.48}{##1}}}
\@namedef{PY@tok@cm}{\let\PY@it=\textit\def\PY@tc##1{\textcolor[rgb]{0.24,0.48,0.48}{##1}}}
\@namedef{PY@tok@cpf}{\let\PY@it=\textit\def\PY@tc##1{\textcolor[rgb]{0.24,0.48,0.48}{##1}}}
\@namedef{PY@tok@c1}{\let\PY@it=\textit\def\PY@tc##1{\textcolor[rgb]{0.24,0.48,0.48}{##1}}}
\@namedef{PY@tok@cs}{\let\PY@it=\textit\def\PY@tc##1{\textcolor[rgb]{0.24,0.48,0.48}{##1}}}

\def\PYZbs{\char`\\}
\def\PYZus{\char`\_}
\def\PYZob{\char`\{}
\def\PYZcb{\char`\}}
\def\PYZca{\char`\^}
\def\PYZam{\char`\&}
\def\PYZlt{\char`\<}
\def\PYZgt{\char`\>}
\def\PYZsh{\char`\#}
\def\PYZpc{\char`\%}
\def\PYZdl{\char`\$}
\def\PYZhy{\char`\-}
\def\PYZsq{\char`\'}
\def\PYZdq{\char`\"}
\def\PYZti{\char`\~}
% for compatibility with earlier versions
\def\PYZat{@}
\def\PYZlb{[}
\def\PYZrb{]}
\makeatother


    % For linebreaks inside Verbatim environment from package fancyvrb.
    \makeatletter
        \newbox\Wrappedcontinuationbox
        \newbox\Wrappedvisiblespacebox
        \newcommand*\Wrappedvisiblespace {\textcolor{red}{\textvisiblespace}}
        \newcommand*\Wrappedcontinuationsymbol {\textcolor{red}{\llap{\tiny$\m@th\hookrightarrow$}}}
        \newcommand*\Wrappedcontinuationindent {3ex }
        \newcommand*\Wrappedafterbreak {\kern\Wrappedcontinuationindent\copy\Wrappedcontinuationbox}
        % Take advantage of the already applied Pygments mark-up to insert
        % potential linebreaks for TeX processing.
        %        {, <, #, %, $, ' and ": go to next line.
        %        _, }, ^, &, >, - and ~: stay at end of broken line.
        % Use of \textquotesingle for straight quote.
        \newcommand*\Wrappedbreaksatspecials {%
            \def\PYGZus{\discretionary{\char`\_}{\Wrappedafterbreak}{\char`\_}}%
            \def\PYGZob{\discretionary{}{\Wrappedafterbreak\char`\{}{\char`\{}}%
            \def\PYGZcb{\discretionary{\char`\}}{\Wrappedafterbreak}{\char`\}}}%
            \def\PYGZca{\discretionary{\char`\^}{\Wrappedafterbreak}{\char`\^}}%
            \def\PYGZam{\discretionary{\char`\&}{\Wrappedafterbreak}{\char`\&}}%
            \def\PYGZlt{\discretionary{}{\Wrappedafterbreak\char`\<}{\char`\<}}%
            \def\PYGZgt{\discretionary{\char`\>}{\Wrappedafterbreak}{\char`\>}}%
            \def\PYGZsh{\discretionary{}{\Wrappedafterbreak\char`\#}{\char`\#}}%
            \def\PYGZpc{\discretionary{}{\Wrappedafterbreak\char`\%}{\char`\%}}%
            \def\PYGZdl{\discretionary{}{\Wrappedafterbreak\char`\$}{\char`\$}}%
            \def\PYGZhy{\discretionary{\char`\-}{\Wrappedafterbreak}{\char`\-}}%
            \def\PYGZsq{\discretionary{}{\Wrappedafterbreak\textquotesingle}{\textquotesingle}}%
            \def\PYGZdq{\discretionary{}{\Wrappedafterbreak\char`\"}{\char`\"}}%
            \def\PYGZti{\discretionary{\char`\~}{\Wrappedafterbreak}{\char`\~}}%
        }
        % Some characters . , ; ? ! / are not pygmentized.
        % This macro makes them "active" and they will insert potential linebreaks
        \newcommand*\Wrappedbreaksatpunct {%
            \lccode`\~`\.\lowercase{\def~}{\discretionary{\hbox{\char`\.}}{\Wrappedafterbreak}{\hbox{\char`\.}}}%
            \lccode`\~`\,\lowercase{\def~}{\discretionary{\hbox{\char`\,}}{\Wrappedafterbreak}{\hbox{\char`\,}}}%
            \lccode`\~`\;\lowercase{\def~}{\discretionary{\hbox{\char`\;}}{\Wrappedafterbreak}{\hbox{\char`\;}}}%
            \lccode`\~`\:\lowercase{\def~}{\discretionary{\hbox{\char`\:}}{\Wrappedafterbreak}{\hbox{\char`\:}}}%
            \lccode`\~`\?\lowercase{\def~}{\discretionary{\hbox{\char`\?}}{\Wrappedafterbreak}{\hbox{\char`\?}}}%
            \lccode`\~`\!\lowercase{\def~}{\discretionary{\hbox{\char`\!}}{\Wrappedafterbreak}{\hbox{\char`\!}}}%
            \lccode`\~`\/\lowercase{\def~}{\discretionary{\hbox{\char`\/}}{\Wrappedafterbreak}{\hbox{\char`\/}}}%
            \catcode`\.\active
            \catcode`\,\active
            \catcode`\;\active
            \catcode`\:\active
            \catcode`\?\active
            \catcode`\!\active
            \catcode`\/\active
            \lccode`\~`\~
        }
    \makeatother

    \let\OriginalVerbatim=\Verbatim
    \makeatletter
    \renewcommand{\Verbatim}[1][1]{%
        %\parskip\z@skip
        \sbox\Wrappedcontinuationbox {\Wrappedcontinuationsymbol}%
        \sbox\Wrappedvisiblespacebox {\FV@SetupFont\Wrappedvisiblespace}%
        \def\FancyVerbFormatLine ##1{\hsize\linewidth
            \vtop{\raggedright\hyphenpenalty\z@\exhyphenpenalty\z@
                \doublehyphendemerits\z@\finalhyphendemerits\z@
                \strut ##1\strut}%
        }%
        % If the linebreak is at a space, the latter will be displayed as visible
        % space at end of first line, and a continuation symbol starts next line.
        % Stretch/shrink are however usually zero for typewriter font.
        \def\FV@Space {%
            \nobreak\hskip\z@ plus\fontdimen3\font minus\fontdimen4\font
            \discretionary{\copy\Wrappedvisiblespacebox}{\Wrappedafterbreak}
            {\kern\fontdimen2\font}%
        }%

        % Allow breaks at special characters using \PYG... macros.
        \Wrappedbreaksatspecials
        % Breaks at punctuation characters . , ; ? ! and / need catcode=\active
        \OriginalVerbatim[#1,codes*=\Wrappedbreaksatpunct]%
    }
    \makeatother

    % Exact colors from NB
    \definecolor{incolor}{HTML}{303F9F}
    \definecolor{outcolor}{HTML}{D84315}
    \definecolor{cellborder}{HTML}{CFCFCF}
    \definecolor{cellbackground}{HTML}{F7F7F7}

    % prompt
    \makeatletter
    \newcommand{\boxspacing}{\kern\kvtcb@left@rule\kern\kvtcb@boxsep}
    \makeatother
    \newcommand{\prompt}[4]{
        {\ttfamily\llap{{\color{#2}[#3]:\hspace{3pt}#4}}\vspace{-\baselineskip}}
    }
    

    
    % Prevent overflowing lines due to hard-to-break entities
    \sloppy
    % Setup hyperref package
    \hypersetup{
      breaklinks=true,  % so long urls are correctly broken across lines
      colorlinks=true,
      urlcolor=urlcolor,
      linkcolor=linkcolor,
      citecolor=citecolor,
      }
    % Slightly bigger margins than the latex defaults
    
    \geometry{verbose,tmargin=1in,bmargin=1in,lmargin=1in,rmargin=1in}
    
    

\begin{document}
    
    \maketitle
    
    

    
    \section{Homework 2 (Due 31 Jan)}\label{homework-2-due-31-jan}

\textbf{Due January 31 (midnight)}

Total points: \textbf{100}.

    \textbf{Practicalities about homeworks and projects}

\begin{enumerate}
\def\labelenumi{\arabic{enumi}.}
\tightlist
\item
  You can work in groups (optimal groups are often 2-3 people) or by
  yourself. If you work as a group you can hand in one answer only if
  you wish. \textbf{Remember to write your name(s)}!
\item
  Homeworks are available approximately ten days before the deadline.
  You should anticipate this work.
\item
  How do I(we) hand in? You can hand in the paper and pencil exercises
  as a \textbf{single scanned PDF document}. For this homework this
  applies to exercises 1-5. Your jupyter notebook file should be
  converted to a \textbf{PDF} file, attached to the same PDF file as for
  the pencil and paper exercises. All files should be uploaded to
  Gradescope.
\end{enumerate}

\textbf{\href{../resources/gradescope-submissions.md}{Instructions for
submitting to Gradescope}.}

    \subsubsection{Exercise 1 (10 pt), Forces, discussion questions, test
your
intuition}\label{exercise-1-10-pt-forces-discussion-questions-test-your-intuition}

These questions expect not only an answer, but an explanation of your
reasoning.

\emph{To receive full credit, these answers should include both the
underlying physics that explains your answer, but how you feel about
that answer (i.e.~are you confident? do you like this answer? do it
unsettle you? it's ok to feel uncomfortable right now with these ideas;
physics intuition is developed and often has to be resolved with our
everyday experiences).}

\begin{itemize}
\item
  1a (2pt) Single force. Can an object affected only by a single force
  have zero acceleration?
\item
  1b (2pt) Zero velocity. If you throw a ball vertically it has zero
  velocity at its maximum point. Does it also have zero acceleration at
  this point?
\item
  1c (3pt) Acceleration of gravity. You measure the acceleration of
  gravity in an elevator moving at a velocity of 9.8m/s downwards. What
  will you measure?
\item
  1d (3pt) Air resistance. You throw a ball straight up and measure the
  velocity as it passes you on its way down. Will the velocity be
  larger, the same, or smaller if you did the same experiment in vacuum?
\end{itemize}

    \subsubsection{Exercise 2 (10 pt), setting up forces, Newton's second
law}\label{exercise-2-10-pt-setting-up-forces-newtons-second-law}

Useful material here to read is 1. Taylor chapters 1.3 and 1.4 and

\begin{enumerate}
\def\labelenumi{\arabic{enumi}.}
\setcounter{enumi}{1}
\tightlist
\item
  Malthe-Sørenssen chapters 5.1, 5.2 and 5.3
\end{enumerate}

A person jumps from an airplane, falling freely for several seconds
before the person pulls the cord of the parachute and the parachute
unfolds. * 2a (3pt) Identify the forces acting on the parachuter and
draw a free-body diagram of the parachuter before the person has pulled
the cord. Include a brief discussion of any assumptions you make,
motivate and justify your choices.

\begin{itemize}
\item
  2b (3pt) Identify the forces acting on the parachuter and draw a
  free-body diagram of the parachuter after the person has pulled the
  cord. Include a brief discussion of any assumptions you make, motivate
  and justify your choices.
\item
  2c (4pt) Sketch the net force acting on the parachuter as a function
  of time, F(t). Your sketch should be qualitatively correct, indicate
  the axes, and show clearly which forces are acting on the parachuter
  before and after the person has pulled the cord.
\end{itemize}

    \subsubsection{Exercise 3 (10 pt), Space shuttle with air
resistance}\label{exercise-3-10-pt-space-shuttle-with-air-resistance}

Useful material here to read is: Malthe-Sørenssen chapters 5.1, 5.2 and
5.3

During lift-off of the space shuttle the engines provide a force of
\(35\times 10^{6}\) N. The mass of the shuttle is approximately
\(2\times 10^6\) kg.

\begin{itemize}
\item
  3a (3pt) Draw a free-body diagram of the space shuttle immediately
  after lift-off.
\item
  3b (3pt) Find an expression for the acceleration of the space shuttle
  immediately after lift-off.
\end{itemize}

Let us assume that the force from the engines is constant, and that the
mass of the space shuttle does not change significantly over the first
20 s. * 3c (4pt) Find the velocity and position of the space shuttle
after 20 s if you ignore air resistance.

    \subsubsection{Exercise 4 (15 pt), Hitting a golf
ball}\label{exercise-4-15-pt-hitting-a-golf-ball}

Useful material here to read is * Taylor chapters 1.3-1.6 and *
Malthe-Sørenssen chapter 6.3-6.4 and 7.1-7.3

\textbf{Do Taylor exercise 1.35}. The formulae you obtain here will be
useful for the numerical exercises below (see exercise 6 below).

\emph{Repeated below consistent with
\href{https://en.wikipedia.org/wiki/Fair_use}{fair use practices}}

A golf ball is hit from ground level with a speed \(v_0\) in a direction
due east that is at an angle \(\theta\) above the horizontal.

\begin{itemize}
\tightlist
\item
  4a (5 pt) Neglecting air resistance, use Newton's second law to find
  the position as a function of time, using the coordinates \(x\)
  measured east, \(y\) measured north, and \(z\) measured up.
\item
  4b (5 pt) Find the time the golf ball is in the air and how far it
  travels in that time.
\end{itemize}

\textbf{Reflect on the form of your answer.}

\begin{itemize}
\tightlist
\item
  4c (5 pt) Your answers should depend on \(v_0\) and \(\theta\). What
  can you say about the dependence of the time the golf ball is in the
  air and the distance it travels on these two variables? How can we
  believe this functional form of your answer? How can we check it?
  Propose a check and check that your answer is consistent with this
  check.
\end{itemize}

    \subsubsection{Exercise 5 (15 pt), ball thrown along a sloped
ramp}\label{exercise-5-15-pt-ball-thrown-along-a-sloped-ramp}

\textbf{Taylor exercise 1.39.} Make sure to draw your setup clearly,
show your free-body diagram, and explain any assumptions you make to
solve the problem.

\emph{Repeated below consistent with
\href{https://en.wikipedia.org/wiki/Fair_use}{fair use practices}}

A ball is thrown with initial speed \(v_0\) up an inclined plane. The
plane is inclined at an angle \(\phi\) above the horizontal, and the
ball's initial velocity is at an angle \(\theta\) above the plane.
Choose axes with \(x\) measured up the slope, \(y\) normal to the slope,
and \(z\) across it.

\begin{itemize}
\tightlist
\item
  5a (5 pt) Write down Newton's second law using these axes and find the
  ball's position as a function of time. \textbf{Make sure to include
  the FBD and any assumptions you make.}
\item
  5b (5 pt) Show that the ball lands a distance
\end{itemize}

\[R=2v_0^2\dfrac{\sin\theta\cos\left(\theta + \phi\right)}{g \cos^2 \phi}\]

from its launch point. \textbf{This is measured up the ramp (i.e., along
it).}

\begin{itemize}
\tightlist
\item
  5c (5 pt) Show that for given \(v_0\) and \(\phi\), the maximum range
  up the inclined plane is:
\end{itemize}

\[R_{\text{max}}=\dfrac{v_0^2}{g(1+\sin\phi)}\]

    \subsubsection{Exercise 6 (40pt), Numerical elements, moving to more
than one
dimension}\label{exercise-6-40pt-numerical-elements-moving-to-more-than-one-dimension}

\textbf{This exercise should be handed in as a jupyter-notebook} at D2L.
Remember to write your name(s).

Last week we: 1. Analytically mapped 1D motion over some time

\begin{enumerate}
\def\labelenumi{\arabic{enumi}.}
\setcounter{enumi}{1}
\item
  Gained practice with functions
\item
  Reviewed vectors and matrices in Python
\end{enumerate}

This week we will: 1. Practice using Python syntax and variable
manipulation

\begin{enumerate}
\def\labelenumi{\arabic{enumi}.}
\setcounter{enumi}{1}
\item
  Utilize analytical solutions to create more refined functions
\item
  Work in two, three or even higher dimensions
\end{enumerate}

This material will then serve as background for the numerical part of
homework 3. The first part is a simple warm-up, with hints and
suggestions you can use for the code to write below.

    \begin{tcolorbox}[breakable, size=fbox, boxrule=1pt, pad at break*=1mm,colback=cellbackground, colframe=cellborder]
\prompt{In}{incolor}{2}{\boxspacing}
\begin{Verbatim}[commandchars=\\\{\}]
\PY{o}{\PYZpc{}}\PY{k}{matplotlib} inline

\PY{c+c1}{\PYZsh{} As usual, here are some useful packages we will be using. Feel free to use more and experiment as you wish.}

\PY{k+kn}{import}\PY{+w}{ }\PY{n+nn}{numpy}\PY{+w}{ }\PY{k}{as}\PY{+w}{ }\PY{n+nn}{np}
\PY{k+kn}{import}\PY{+w}{ }\PY{n+nn}{matplotlib}\PY{n+nn}{.}\PY{n+nn}{pyplot}\PY{+w}{ }\PY{k}{as}\PY{+w}{ }\PY{n+nn}{plt}
\PY{k+kn}{from}\PY{+w}{ }\PY{n+nn}{mpl\PYZus{}toolkits}\PY{+w}{ }\PY{k+kn}{import} \PY{n}{mplot3d}
\PY{o}{\PYZpc{}}\PY{k}{matplotlib} inline
\end{Verbatim}
\end{tcolorbox}

    In class (the falling baseball example) we used an analytical expression
for the height of a falling ball. In the first homework we used instead
the position from experiment (Usain Bolt's 100m record run) and stored
this information with one-dimensional arrays in Python.

Let us get some practice with this. The cell below creates two arrays,
one containing the times to be analyzed and the other containing the
\(x\) and \(y\) components of the position vector at each point in time.
This is a two-dimensional object. The second array is initially empty.
Then we define the initial position to be \(x=2\) and \(y=1\). Take a
look at the code and comments to get an understanding of what is
happening. Feel free to play around with it.

    \begin{tcolorbox}[breakable, size=fbox, boxrule=1pt, pad at break*=1mm,colback=cellbackground, colframe=cellborder]
\prompt{In}{incolor}{3}{\boxspacing}
\begin{Verbatim}[commandchars=\\\{\}]
\PY{n}{tf} \PY{o}{=} \PY{l+m+mi}{4} \PY{c+c1}{\PYZsh{}length of value to be analyzed}
\PY{n}{dt} \PY{o}{=} \PY{l+m+mf}{.001} \PY{c+c1}{\PYZsh{} step sizes}
\PY{n}{t} \PY{o}{=} \PY{n}{np}\PY{o}{.}\PY{n}{arange}\PY{p}{(}\PY{l+m+mf}{0.0}\PY{p}{,}\PY{n}{tf}\PY{p}{,}\PY{n}{dt}\PY{p}{)} \PY{c+c1}{\PYZsh{} Creates an evenly spaced time array going from 0 to 3.999, with step sizes .001}
\PY{n}{p} \PY{o}{=} \PY{n}{np}\PY{o}{.}\PY{n}{zeros}\PY{p}{(}\PY{p}{(}\PY{n+nb}{len}\PY{p}{(}\PY{n}{t}\PY{p}{)}\PY{p}{,} \PY{l+m+mi}{2}\PY{p}{)}\PY{p}{)} \PY{c+c1}{\PYZsh{} Creates an empty array of [x,y] arrays (our vectors). Array size is same as the one for time.}
\PY{n}{p}\PY{p}{[}\PY{l+m+mi}{0}\PY{p}{]} \PY{o}{=} \PY{p}{[}\PY{l+m+mf}{2.0}\PY{p}{,}\PY{l+m+mf}{1.0}\PY{p}{]} \PY{c+c1}{\PYZsh{} This sets the inital position to be x = 2 and y = 1}
\end{Verbatim}
\end{tcolorbox}

    Below we are printing specific values of our array to see what is being
stored where. The first number in the array \(r[]\) represents which
array iteration we are looking at, while the number after the represents
which listed number in the array iteration we are getting back.

    \begin{tcolorbox}[breakable, size=fbox, boxrule=1pt, pad at break*=1mm,colback=cellbackground, colframe=cellborder]
\prompt{In}{incolor}{4}{\boxspacing}
\begin{Verbatim}[commandchars=\\\{\}]
\PY{n+nb}{print}\PY{p}{(}\PY{n}{p}\PY{p}{[}\PY{l+m+mi}{0}\PY{p}{]}\PY{p}{)} \PY{c+c1}{\PYZsh{} Prints the first array}
\PY{n+nb}{print}\PY{p}{(}\PY{n}{p}\PY{p}{[}\PY{l+m+mi}{0}\PY{p}{,}\PY{p}{:}\PY{p}{]}\PY{p}{)} \PY{c+c1}{\PYZsh{} Same as above, these commands are interchangeable}
\end{Verbatim}
\end{tcolorbox}

    \begin{Verbatim}[commandchars=\\\{\}]
[2. 1.]
[2. 1.]
    \end{Verbatim}

    \begin{tcolorbox}[breakable, size=fbox, boxrule=1pt, pad at break*=1mm,colback=cellbackground, colframe=cellborder]
\prompt{In}{incolor}{5}{\boxspacing}
\begin{Verbatim}[commandchars=\\\{\}]
\PY{n+nb}{print}\PY{p}{(}\PY{n}{p}\PY{p}{[}\PY{l+m+mi}{3999}\PY{p}{]}\PY{p}{)} \PY{c+c1}{\PYZsh{} Prints the 4000th array}
\end{Verbatim}
\end{tcolorbox}

    \begin{Verbatim}[commandchars=\\\{\}]
[0. 0.]
    \end{Verbatim}

    \begin{tcolorbox}[breakable, size=fbox, boxrule=1pt, pad at break*=1mm,colback=cellbackground, colframe=cellborder]
\prompt{In}{incolor}{6}{\boxspacing}
\begin{Verbatim}[commandchars=\\\{\}]
\PY{n+nb}{print}\PY{p}{(}\PY{n}{p}\PY{p}{[}\PY{l+m+mi}{0}\PY{p}{,}\PY{l+m+mi}{0}\PY{p}{]}\PY{p}{)} \PY{c+c1}{\PYZsh{} Prints the first value of the first array}
\end{Verbatim}
\end{tcolorbox}

    \begin{Verbatim}[commandchars=\\\{\}]
2.0
    \end{Verbatim}

    \begin{tcolorbox}[breakable, size=fbox, boxrule=1pt, pad at break*=1mm,colback=cellbackground, colframe=cellborder]
\prompt{In}{incolor}{7}{\boxspacing}
\begin{Verbatim}[commandchars=\\\{\}]
\PY{n+nb}{print}\PY{p}{(}\PY{n}{p}\PY{p}{[}\PY{l+m+mi}{0}\PY{p}{,}\PY{l+m+mi}{1}\PY{p}{]}\PY{p}{)} \PY{c+c1}{\PYZsh{} Prints the second value of first array}
\PY{n+nb}{print}\PY{p}{(}\PY{n}{p}\PY{p}{[}\PY{p}{:}\PY{p}{,}\PY{l+m+mi}{0}\PY{p}{]}\PY{p}{)} \PY{c+c1}{\PYZsh{} Prints the first value of all the arrays}
\end{Verbatim}
\end{tcolorbox}

    \begin{Verbatim}[commandchars=\\\{\}]
1.0
[2. 0. 0. {\ldots} 0. 0. 0.]
    \end{Verbatim}

    Then try running this cell. Notice how it gives an error since we did
not implement a third dimension into our arrays

    \begin{tcolorbox}[breakable, size=fbox, boxrule=1pt, pad at break*=1mm,colback=cellbackground, colframe=cellborder]
\prompt{In}{incolor}{10}{\boxspacing}
\begin{Verbatim}[commandchars=\\\{\}]
\PY{c+c1}{\PYZsh{}print(p[:,2])}
\end{Verbatim}
\end{tcolorbox}

    In the cell below we want to manipulate the arrays. In this example we
make each vector's \(x\) component valued the same as their respective
vector's position in the iteration and the \(y\) value will be twice
that value, except for the first vector, which we have already set. That
is we have
\(p[0] = [2,1], p[1] = [1,2], p[2] = [2,4], p[3] = [3,6], ...\)

Here we set up an array for \(x\) and \(y\) values.

    \begin{tcolorbox}[breakable, size=fbox, boxrule=1pt, pad at break*=1mm,colback=cellbackground, colframe=cellborder]
\prompt{In}{incolor}{14}{\boxspacing}
\begin{Verbatim}[commandchars=\\\{\}]
\PY{k}{for} \PY{n}{i} \PY{o+ow}{in} \PY{n+nb}{range}\PY{p}{(}\PY{l+m+mi}{1}\PY{p}{,}\PY{l+m+mi}{3999}\PY{p}{)}\PY{p}{:}
    \PY{n}{p}\PY{p}{[}\PY{n}{i}\PY{p}{]} \PY{o}{=} \PY{p}{[}\PY{n}{i}\PY{p}{,}\PY{l+m+mi}{2}\PY{o}{*}\PY{n}{i}\PY{p}{]}
\PY{c+c1}{\PYZsh{} Checker cell to make sure your code is performing correctly}
\PY{n}{c} \PY{o}{=} \PY{l+m+mi}{0}
\PY{k}{for} \PY{n}{i} \PY{o+ow}{in} \PY{n+nb}{range}\PY{p}{(}\PY{l+m+mi}{0}\PY{p}{,}\PY{l+m+mi}{3999}\PY{p}{)}\PY{p}{:}
    \PY{k}{if} \PY{n}{i} \PY{o}{==} \PY{l+m+mi}{0}\PY{p}{:}
        \PY{k}{if} \PY{n}{p}\PY{p}{[}\PY{n}{i}\PY{p}{,}\PY{l+m+mi}{0}\PY{p}{]} \PY{o}{!=} \PY{l+m+mf}{2.0}\PY{p}{:}
            \PY{n}{c} \PY{o}{+}\PY{o}{=} \PY{l+m+mi}{1}
        \PY{k}{if} \PY{n}{p}\PY{p}{[}\PY{n}{i}\PY{p}{,}\PY{l+m+mi}{1}\PY{p}{]} \PY{o}{!=} \PY{l+m+mf}{1.0}\PY{p}{:}
            \PY{n}{c} \PY{o}{+}\PY{o}{=} \PY{l+m+mi}{1}
    \PY{k}{else}\PY{p}{:}
        \PY{k}{if} \PY{n}{p}\PY{p}{[}\PY{n}{i}\PY{p}{,}\PY{l+m+mi}{0}\PY{p}{]} \PY{o}{!=} \PY{l+m+mf}{1.0}\PY{o}{*}\PY{n}{i}\PY{p}{:}
            \PY{n}{c} \PY{o}{+}\PY{o}{=} \PY{l+m+mi}{1}
        \PY{k}{if} \PY{n}{p}\PY{p}{[}\PY{n}{i}\PY{p}{,}\PY{l+m+mi}{1}\PY{p}{]} \PY{o}{!=} \PY{l+m+mf}{2.0}\PY{o}{*}\PY{n}{i}\PY{p}{:}
            \PY{n}{c} \PY{o}{+}\PY{o}{=} \PY{l+m+mi}{1}

\PY{k}{if} \PY{n}{c} \PY{o}{==} \PY{l+m+mi}{0}\PY{p}{:}
    \PY{n+nb}{print}\PY{p}{(}\PY{l+s+s2}{\PYZdq{}}\PY{l+s+s2}{Success!}\PY{l+s+s2}{\PYZdq{}}\PY{p}{)}
\PY{k}{else}\PY{p}{:}
    \PY{n+nb}{print}\PY{p}{(}\PY{l+s+s2}{\PYZdq{}}\PY{l+s+s2}{There is an error in your code}\PY{l+s+s2}{\PYZdq{}}\PY{p}{)}
\end{Verbatim}
\end{tcolorbox}

    \begin{Verbatim}[commandchars=\\\{\}]
Success!
    \end{Verbatim}

    You could also think of an alternative way of storing the above
information. Feel free to explore how to store multidimensional objects.

Last week we studied Usain Bolt's 100m run and in class we studied a
falling baseball. We made basic plots of the baseball moving in one
dimension. This week we will be working with a three-dimensional
variant. This will be useful for our next homeworks and numerical
projects.

Assume we have a soccer ball moving in three dimensions with the
following trajectory:

\[x(t) = 10t\cos{45^{\circ}}\]

\[y(t) = 10t\sin{45^{\circ}}\]

\[z(t) = 10t-\dfrac{9.81}{2}t^2\]

Now let us create a three-dimensional (3D) plot using these equations.
In the cell below we write the equations into their respective labels.
We fix a final time in the code below.

Important Concept: Numpy comes with many mathematical packages, some of
them being the trigonometric functions sine, cosine, tangent. We are
going to utilize these this week. Additionally, these functions work
with radians, so we will also be using a function from Numpy that
converts degrees to radians.

    \begin{tcolorbox}[breakable, size=fbox, boxrule=1pt, pad at break*=1mm,colback=cellbackground, colframe=cellborder]
\prompt{In}{incolor}{11}{\boxspacing}
\begin{Verbatim}[commandchars=\\\{\}]
\PY{n}{tf} \PY{o}{=} \PY{l+m+mf}{2.04}  \PY{c+c1}{\PYZsh{} The final time to be evaluated}
\PY{n}{dt} \PY{o}{=} \PY{l+m+mf}{0.1}  \PY{c+c1}{\PYZsh{} The time step size}
\PY{n}{t} \PY{o}{=} \PY{n}{np}\PY{o}{.}\PY{n}{arange}\PY{p}{(}\PY{l+m+mi}{0}\PY{p}{,}\PY{n}{tf}\PY{p}{,}\PY{n}{dt}\PY{p}{)} \PY{c+c1}{\PYZsh{} The time array}
\PY{n}{theta\PYZus{}deg} \PY{o}{=} \PY{l+m+mi}{45} \PY{c+c1}{\PYZsh{} Degrees}
\PY{n}{theta\PYZus{}rad} \PY{o}{=} \PY{n}{np}\PY{o}{.}\PY{n}{radians}\PY{p}{(}\PY{n}{theta\PYZus{}deg}\PY{p}{)} \PY{c+c1}{\PYZsh{} Converts degrees to their radian counterparts}
\PY{n}{x} \PY{o}{=} \PY{l+m+mi}{10}\PY{o}{*}\PY{n}{t}\PY{o}{*}\PY{n}{np}\PY{o}{.}\PY{n}{cos}\PY{p}{(}\PY{n}{theta\PYZus{}rad}\PY{p}{)} \PY{c+c1}{\PYZsh{} Equation for our x component, utilizing np.cos() and our calculated radians}
\PY{n}{y} \PY{o}{=} \PY{l+m+mi}{10}\PY{o}{*}\PY{n}{t}\PY{o}{*}\PY{n}{np}\PY{o}{.}\PY{n}{sin}\PY{p}{(}\PY{n}{theta\PYZus{}rad}\PY{p}{)} \PY{c+c1}{\PYZsh{} Put the y equation here}
\PY{n}{z} \PY{o}{=} \PY{l+m+mi}{10}\PY{o}{*}\PY{n}{t}\PY{o}{\PYZhy{}}\PY{l+m+mf}{9.81}\PY{o}{/}\PY{l+m+mi}{2}\PY{o}{*}\PY{n}{t}\PY{o}{*}\PY{o}{*}\PY{l+m+mi}{2}\PY{c+c1}{\PYZsh{} Put the z equation here}
\end{Verbatim}
\end{tcolorbox}

    Then we plot it

    \begin{tcolorbox}[breakable, size=fbox, boxrule=1pt, pad at break*=1mm,colback=cellbackground, colframe=cellborder]
\prompt{In}{incolor}{16}{\boxspacing}
\begin{Verbatim}[commandchars=\\\{\}]
\PY{c+c1}{\PYZsh{}\PYZsh{} Once you have entered the proper equations in the cell above, run this cell to plot in 3D}
\PY{n}{fig} \PY{o}{=} \PY{n}{plt}\PY{o}{.}\PY{n}{axes}\PY{p}{(}\PY{n}{projection}\PY{o}{=}\PY{l+s+s1}{\PYZsq{}}\PY{l+s+s1}{3d}\PY{l+s+s1}{\PYZsq{}}\PY{p}{)}
\PY{n}{fig}\PY{o}{.}\PY{n}{set\PYZus{}xlabel}\PY{p}{(}\PY{l+s+s1}{\PYZsq{}}\PY{l+s+s1}{x}\PY{l+s+s1}{\PYZsq{}}\PY{p}{)}
\PY{n}{fig}\PY{o}{.}\PY{n}{set\PYZus{}ylabel}\PY{p}{(}\PY{l+s+s1}{\PYZsq{}}\PY{l+s+s1}{y}\PY{l+s+s1}{\PYZsq{}}\PY{p}{)}
\PY{n}{fig}\PY{o}{.}\PY{n}{set\PYZus{}zlabel}\PY{p}{(}\PY{l+s+s1}{\PYZsq{}}\PY{l+s+s1}{z}\PY{l+s+s1}{\PYZsq{}}\PY{p}{)}
\PY{n}{fig}\PY{o}{.}\PY{n}{scatter}\PY{p}{(}\PY{n}{x}\PY{p}{,}\PY{n}{y}\PY{p}{,}\PY{n}{z}\PY{p}{)}
\end{Verbatim}
\end{tcolorbox}

            \begin{tcolorbox}[breakable, size=fbox, boxrule=.5pt, pad at break*=1mm, opacityfill=0]
\prompt{Out}{outcolor}{16}{\boxspacing}
\begin{Verbatim}[commandchars=\\\{\}]
<mpl\_toolkits.mplot3d.art3d.Path3DCollection at 0x118aa4740>
\end{Verbatim}
\end{tcolorbox}
        
    \begin{itemize}
\tightlist
\item
  6a (8pt) How would you express \(x(t)\), \(y(t)\), \(z(t)\) for this
  problem as a single vector, \(\boldsymbol{r}(t)\)?
\end{itemize}

Then run the code and plot using the array \(r\)

    \begin{tcolorbox}[breakable, size=fbox, boxrule=1pt, pad at break*=1mm,colback=cellbackground, colframe=cellborder]
\prompt{In}{incolor}{17}{\boxspacing}
\begin{Verbatim}[commandchars=\\\{\}]
\PY{c+c1}{\PYZsh{}\PYZsh{} Run this code to plot using our r array }
\PY{c+c1}{\PYZsh{} fig = plt.axes(projection=\PYZsq{}3d\PYZsq{})}
\PY{c+c1}{\PYZsh{} fig.set\PYZus{}xlabel(\PYZsq{}x\PYZsq{})}
\PY{c+c1}{\PYZsh{} fig.set\PYZus{}ylabel(\PYZsq{}y\PYZsq{})}
\PY{c+c1}{\PYZsh{} fig.set\PYZus{}zlabel(\PYZsq{}z\PYZsq{})}
\PY{c+c1}{\PYZsh{} fig.scatter(r[0],r[1],r[2])}
\end{Verbatim}
\end{tcolorbox}

    \begin{itemize}
\tightlist
\item
  6b (8pt) What do you think the benefits and/or disadvantages are from
  expressing our three equations as a single array/vector? This can be
  both from a computational and physics stand point. Use the
  \textbf{Numpy} package to also print the maximum \(x\), \(y\) and
  \(z\) components from \(\boldsymbol{r}\).
\end{itemize}

Complete Exercise 4 above (Taylor exercise 1.35) before moving further.
(Recall that the golf ball was hit due east at an angle \(\theta\) with
respect to the horizontal, and the coordinate directions are \(x\)
measured east, \(y\) north, and \(z\) vertically up.)

\begin{itemize}
\item
  6c (8pt) What is the analytical solution for our theoretical golf
  ball's position \(\boldsymbol{r}(t)\) over time from Exercise 4? Also
  what is the formula for the time \(t_f\) when the golf ball hits the
  ground? Use this to develop a program with a function called for
  example Golfball that utilizes our analytical solutions. This program
  should take in an initial velocity and the angle \(\theta\) that the
  golfball was hit with in degrees. It should also produce a 3D graph of
  the motion. You need also to find the maximum values for \(x\), \(y\)
  and \(z\).
\item
  6d (8pt) Given initial values of \(v_i = 90 m/s\),
  \(\theta = 30^{\circ}\), what would our maximum x, y and z components
  be?
\item
  6e (8pt) Given initial values of \(v_i = 45 m/s\),
  \(\theta = 45^{\circ}\), what would our maximum x, y and z components
  be?
\end{itemize}


    % Add a bibliography block to the postdoc
    
    
    
\end{document}
