\documentclass[11pt]{article}

    \usepackage[breakable]{tcolorbox}
    \usepackage{parskip} % Stop auto-indenting (to mimic markdown behaviour)
    

    % Basic figure setup, for now with no caption control since it's done
    % automatically by Pandoc (which extracts ![](path) syntax from Markdown).
    \usepackage{graphicx}
    % Keep aspect ratio if custom image width or height is specified
    \setkeys{Gin}{keepaspectratio}
    % Maintain compatibility with old templates. Remove in nbconvert 6.0
    \let\Oldincludegraphics\includegraphics
    % Ensure that by default, figures have no caption (until we provide a
    % proper Figure object with a Caption API and a way to capture that
    % in the conversion process - todo).
    \usepackage{caption}
    \DeclareCaptionFormat{nocaption}{}
    \captionsetup{format=nocaption,aboveskip=0pt,belowskip=0pt}

    \usepackage{float}
    \floatplacement{figure}{H} % forces figures to be placed at the correct location
    \usepackage{xcolor} % Allow colors to be defined
    \usepackage{enumerate} % Needed for markdown enumerations to work
    \usepackage{geometry} % Used to adjust the document margins
    \usepackage{amsmath} % Equations
    \usepackage{amssymb} % Equations
    \usepackage{textcomp} % defines textquotesingle
    % Hack from http://tex.stackexchange.com/a/47451/13684:
    \AtBeginDocument{%
        \def\PYZsq{\textquotesingle}% Upright quotes in Pygmentized code
    }
    \usepackage{upquote} % Upright quotes for verbatim code
    \usepackage{eurosym} % defines \euro

    \usepackage{iftex}
    \ifPDFTeX
        \usepackage[T1]{fontenc}
        \IfFileExists{alphabeta.sty}{
              \usepackage{alphabeta}
          }{
              \usepackage[mathletters]{ucs}
              \usepackage[utf8x]{inputenc}
          }
    \else
        \usepackage{fontspec}
        \usepackage{unicode-math}
    \fi

    \usepackage{fancyvrb} % verbatim replacement that allows latex
    \usepackage{grffile} % extends the file name processing of package graphics
                         % to support a larger range
    \makeatletter % fix for old versions of grffile with XeLaTeX
    \@ifpackagelater{grffile}{2019/11/01}
    {
      % Do nothing on new versions
    }
    {
      \def\Gread@@xetex#1{%
        \IfFileExists{"\Gin@base".bb}%
        {\Gread@eps{\Gin@base.bb}}%
        {\Gread@@xetex@aux#1}%
      }
    }
    \makeatother
    \usepackage[Export]{adjustbox} % Used to constrain images to a maximum size
    \adjustboxset{max size={0.9\linewidth}{0.9\paperheight}}

    % The hyperref package gives us a pdf with properly built
    % internal navigation ('pdf bookmarks' for the table of contents,
    % internal cross-reference links, web links for URLs, etc.)
    \usepackage{hyperref}
    % The default LaTeX title has an obnoxious amount of whitespace. By default,
    % titling removes some of it. It also provides customization options.
    \usepackage{titling}
    \usepackage{longtable} % longtable support required by pandoc >1.10
    \usepackage{booktabs}  % table support for pandoc > 1.12.2
    \usepackage{array}     % table support for pandoc >= 2.11.3
    \usepackage{calc}      % table minipage width calculation for pandoc >= 2.11.1
    \usepackage[inline]{enumitem} % IRkernel/repr support (it uses the enumerate* environment)
    \usepackage[normalem]{ulem} % ulem is needed to support strikethroughs (\sout)
                                % normalem makes italics be italics, not underlines
    \usepackage{soul}      % strikethrough (\st) support for pandoc >= 3.0.0
    \usepackage{mathrsfs}
    

    
    % Colors for the hyperref package
    \definecolor{urlcolor}{rgb}{0,.145,.698}
    \definecolor{linkcolor}{rgb}{.71,0.21,0.01}
    \definecolor{citecolor}{rgb}{.12,.54,.11}

    % ANSI colors
    \definecolor{ansi-black}{HTML}{3E424D}
    \definecolor{ansi-black-intense}{HTML}{282C36}
    \definecolor{ansi-red}{HTML}{E75C58}
    \definecolor{ansi-red-intense}{HTML}{B22B31}
    \definecolor{ansi-green}{HTML}{00A250}
    \definecolor{ansi-green-intense}{HTML}{007427}
    \definecolor{ansi-yellow}{HTML}{DDB62B}
    \definecolor{ansi-yellow-intense}{HTML}{B27D12}
    \definecolor{ansi-blue}{HTML}{208FFB}
    \definecolor{ansi-blue-intense}{HTML}{0065CA}
    \definecolor{ansi-magenta}{HTML}{D160C4}
    \definecolor{ansi-magenta-intense}{HTML}{A03196}
    \definecolor{ansi-cyan}{HTML}{60C6C8}
    \definecolor{ansi-cyan-intense}{HTML}{258F8F}
    \definecolor{ansi-white}{HTML}{C5C1B4}
    \definecolor{ansi-white-intense}{HTML}{A1A6B2}
    \definecolor{ansi-default-inverse-fg}{HTML}{FFFFFF}
    \definecolor{ansi-default-inverse-bg}{HTML}{000000}

    % common color for the border for error outputs.
    \definecolor{outerrorbackground}{HTML}{FFDFDF}

    % commands and environments needed by pandoc snippets
    % extracted from the output of `pandoc -s`
    \providecommand{\tightlist}{%
      \setlength{\itemsep}{0pt}\setlength{\parskip}{0pt}}
    \DefineVerbatimEnvironment{Highlighting}{Verbatim}{commandchars=\\\{\}}
    % Add ',fontsize=\small' for more characters per line
    \newenvironment{Shaded}{}{}
    \newcommand{\KeywordTok}[1]{\textcolor[rgb]{0.00,0.44,0.13}{\textbf{{#1}}}}
    \newcommand{\DataTypeTok}[1]{\textcolor[rgb]{0.56,0.13,0.00}{{#1}}}
    \newcommand{\DecValTok}[1]{\textcolor[rgb]{0.25,0.63,0.44}{{#1}}}
    \newcommand{\BaseNTok}[1]{\textcolor[rgb]{0.25,0.63,0.44}{{#1}}}
    \newcommand{\FloatTok}[1]{\textcolor[rgb]{0.25,0.63,0.44}{{#1}}}
    \newcommand{\CharTok}[1]{\textcolor[rgb]{0.25,0.44,0.63}{{#1}}}
    \newcommand{\StringTok}[1]{\textcolor[rgb]{0.25,0.44,0.63}{{#1}}}
    \newcommand{\CommentTok}[1]{\textcolor[rgb]{0.38,0.63,0.69}{\textit{{#1}}}}
    \newcommand{\OtherTok}[1]{\textcolor[rgb]{0.00,0.44,0.13}{{#1}}}
    \newcommand{\AlertTok}[1]{\textcolor[rgb]{1.00,0.00,0.00}{\textbf{{#1}}}}
    \newcommand{\FunctionTok}[1]{\textcolor[rgb]{0.02,0.16,0.49}{{#1}}}
    \newcommand{\RegionMarkerTok}[1]{{#1}}
    \newcommand{\ErrorTok}[1]{\textcolor[rgb]{1.00,0.00,0.00}{\textbf{{#1}}}}
    \newcommand{\NormalTok}[1]{{#1}}

    % Additional commands for more recent versions of Pandoc
    \newcommand{\ConstantTok}[1]{\textcolor[rgb]{0.53,0.00,0.00}{{#1}}}
    \newcommand{\SpecialCharTok}[1]{\textcolor[rgb]{0.25,0.44,0.63}{{#1}}}
    \newcommand{\VerbatimStringTok}[1]{\textcolor[rgb]{0.25,0.44,0.63}{{#1}}}
    \newcommand{\SpecialStringTok}[1]{\textcolor[rgb]{0.73,0.40,0.53}{{#1}}}
    \newcommand{\ImportTok}[1]{{#1}}
    \newcommand{\DocumentationTok}[1]{\textcolor[rgb]{0.73,0.13,0.13}{\textit{{#1}}}}
    \newcommand{\AnnotationTok}[1]{\textcolor[rgb]{0.38,0.63,0.69}{\textbf{\textit{{#1}}}}}
    \newcommand{\CommentVarTok}[1]{\textcolor[rgb]{0.38,0.63,0.69}{\textbf{\textit{{#1}}}}}
    \newcommand{\VariableTok}[1]{\textcolor[rgb]{0.10,0.09,0.49}{{#1}}}
    \newcommand{\ControlFlowTok}[1]{\textcolor[rgb]{0.00,0.44,0.13}{\textbf{{#1}}}}
    \newcommand{\OperatorTok}[1]{\textcolor[rgb]{0.40,0.40,0.40}{{#1}}}
    \newcommand{\BuiltInTok}[1]{{#1}}
    \newcommand{\ExtensionTok}[1]{{#1}}
    \newcommand{\PreprocessorTok}[1]{\textcolor[rgb]{0.74,0.48,0.00}{{#1}}}
    \newcommand{\AttributeTok}[1]{\textcolor[rgb]{0.49,0.56,0.16}{{#1}}}
    \newcommand{\InformationTok}[1]{\textcolor[rgb]{0.38,0.63,0.69}{\textbf{\textit{{#1}}}}}
    \newcommand{\WarningTok}[1]{\textcolor[rgb]{0.38,0.63,0.69}{\textbf{\textit{{#1}}}}}
    \makeatletter
    \newsavebox\pandoc@box
    \newcommand*\pandocbounded[1]{%
      \sbox\pandoc@box{#1}%
      % scaling factors for width and height
      \Gscale@div\@tempa\textheight{\dimexpr\ht\pandoc@box+\dp\pandoc@box\relax}%
      \Gscale@div\@tempb\linewidth{\wd\pandoc@box}%
      % select the smaller of both
      \ifdim\@tempb\p@<\@tempa\p@
        \let\@tempa\@tempb
      \fi
      % scaling accordingly (\@tempa < 1)
      \ifdim\@tempa\p@<\p@
        \scalebox{\@tempa}{\usebox\pandoc@box}%
      % scaling not needed, use as it is
      \else
        \usebox{\pandoc@box}%
      \fi
    }
    \makeatother

    % Define a nice break command that doesn't care if a line doesn't already
    % exist.
    \def\br{\hspace*{\fill} \\* }
    % Math Jax compatibility definitions
    \def\gt{>}
    \def\lt{<}
    \let\Oldtex\TeX
    \let\Oldlatex\LaTeX
    \renewcommand{\TeX}{\textrm{\Oldtex}}
    \renewcommand{\LaTeX}{\textrm{\Oldlatex}}
    % Document parameters
    % Document title
    \title{hw7}
    
    
    
    
    
    
    
% Pygments definitions
\makeatletter
\def\PY@reset{\let\PY@it=\relax \let\PY@bf=\relax%
    \let\PY@ul=\relax \let\PY@tc=\relax%
    \let\PY@bc=\relax \let\PY@ff=\relax}
\def\PY@tok#1{\csname PY@tok@#1\endcsname}
\def\PY@toks#1+{\ifx\relax#1\empty\else%
    \PY@tok{#1}\expandafter\PY@toks\fi}
\def\PY@do#1{\PY@bc{\PY@tc{\PY@ul{%
    \PY@it{\PY@bf{\PY@ff{#1}}}}}}}
\def\PY#1#2{\PY@reset\PY@toks#1+\relax+\PY@do{#2}}

\@namedef{PY@tok@w}{\def\PY@tc##1{\textcolor[rgb]{0.73,0.73,0.73}{##1}}}
\@namedef{PY@tok@c}{\let\PY@it=\textit\def\PY@tc##1{\textcolor[rgb]{0.24,0.48,0.48}{##1}}}
\@namedef{PY@tok@cp}{\def\PY@tc##1{\textcolor[rgb]{0.61,0.40,0.00}{##1}}}
\@namedef{PY@tok@k}{\let\PY@bf=\textbf\def\PY@tc##1{\textcolor[rgb]{0.00,0.50,0.00}{##1}}}
\@namedef{PY@tok@kp}{\def\PY@tc##1{\textcolor[rgb]{0.00,0.50,0.00}{##1}}}
\@namedef{PY@tok@kt}{\def\PY@tc##1{\textcolor[rgb]{0.69,0.00,0.25}{##1}}}
\@namedef{PY@tok@o}{\def\PY@tc##1{\textcolor[rgb]{0.40,0.40,0.40}{##1}}}
\@namedef{PY@tok@ow}{\let\PY@bf=\textbf\def\PY@tc##1{\textcolor[rgb]{0.67,0.13,1.00}{##1}}}
\@namedef{PY@tok@nb}{\def\PY@tc##1{\textcolor[rgb]{0.00,0.50,0.00}{##1}}}
\@namedef{PY@tok@nf}{\def\PY@tc##1{\textcolor[rgb]{0.00,0.00,1.00}{##1}}}
\@namedef{PY@tok@nc}{\let\PY@bf=\textbf\def\PY@tc##1{\textcolor[rgb]{0.00,0.00,1.00}{##1}}}
\@namedef{PY@tok@nn}{\let\PY@bf=\textbf\def\PY@tc##1{\textcolor[rgb]{0.00,0.00,1.00}{##1}}}
\@namedef{PY@tok@ne}{\let\PY@bf=\textbf\def\PY@tc##1{\textcolor[rgb]{0.80,0.25,0.22}{##1}}}
\@namedef{PY@tok@nv}{\def\PY@tc##1{\textcolor[rgb]{0.10,0.09,0.49}{##1}}}
\@namedef{PY@tok@no}{\def\PY@tc##1{\textcolor[rgb]{0.53,0.00,0.00}{##1}}}
\@namedef{PY@tok@nl}{\def\PY@tc##1{\textcolor[rgb]{0.46,0.46,0.00}{##1}}}
\@namedef{PY@tok@ni}{\let\PY@bf=\textbf\def\PY@tc##1{\textcolor[rgb]{0.44,0.44,0.44}{##1}}}
\@namedef{PY@tok@na}{\def\PY@tc##1{\textcolor[rgb]{0.41,0.47,0.13}{##1}}}
\@namedef{PY@tok@nt}{\let\PY@bf=\textbf\def\PY@tc##1{\textcolor[rgb]{0.00,0.50,0.00}{##1}}}
\@namedef{PY@tok@nd}{\def\PY@tc##1{\textcolor[rgb]{0.67,0.13,1.00}{##1}}}
\@namedef{PY@tok@s}{\def\PY@tc##1{\textcolor[rgb]{0.73,0.13,0.13}{##1}}}
\@namedef{PY@tok@sd}{\let\PY@it=\textit\def\PY@tc##1{\textcolor[rgb]{0.73,0.13,0.13}{##1}}}
\@namedef{PY@tok@si}{\let\PY@bf=\textbf\def\PY@tc##1{\textcolor[rgb]{0.64,0.35,0.47}{##1}}}
\@namedef{PY@tok@se}{\let\PY@bf=\textbf\def\PY@tc##1{\textcolor[rgb]{0.67,0.36,0.12}{##1}}}
\@namedef{PY@tok@sr}{\def\PY@tc##1{\textcolor[rgb]{0.64,0.35,0.47}{##1}}}
\@namedef{PY@tok@ss}{\def\PY@tc##1{\textcolor[rgb]{0.10,0.09,0.49}{##1}}}
\@namedef{PY@tok@sx}{\def\PY@tc##1{\textcolor[rgb]{0.00,0.50,0.00}{##1}}}
\@namedef{PY@tok@m}{\def\PY@tc##1{\textcolor[rgb]{0.40,0.40,0.40}{##1}}}
\@namedef{PY@tok@gh}{\let\PY@bf=\textbf\def\PY@tc##1{\textcolor[rgb]{0.00,0.00,0.50}{##1}}}
\@namedef{PY@tok@gu}{\let\PY@bf=\textbf\def\PY@tc##1{\textcolor[rgb]{0.50,0.00,0.50}{##1}}}
\@namedef{PY@tok@gd}{\def\PY@tc##1{\textcolor[rgb]{0.63,0.00,0.00}{##1}}}
\@namedef{PY@tok@gi}{\def\PY@tc##1{\textcolor[rgb]{0.00,0.52,0.00}{##1}}}
\@namedef{PY@tok@gr}{\def\PY@tc##1{\textcolor[rgb]{0.89,0.00,0.00}{##1}}}
\@namedef{PY@tok@ge}{\let\PY@it=\textit}
\@namedef{PY@tok@gs}{\let\PY@bf=\textbf}
\@namedef{PY@tok@ges}{\let\PY@bf=\textbf\let\PY@it=\textit}
\@namedef{PY@tok@gp}{\let\PY@bf=\textbf\def\PY@tc##1{\textcolor[rgb]{0.00,0.00,0.50}{##1}}}
\@namedef{PY@tok@go}{\def\PY@tc##1{\textcolor[rgb]{0.44,0.44,0.44}{##1}}}
\@namedef{PY@tok@gt}{\def\PY@tc##1{\textcolor[rgb]{0.00,0.27,0.87}{##1}}}
\@namedef{PY@tok@err}{\def\PY@bc##1{{\setlength{\fboxsep}{\string -\fboxrule}\fcolorbox[rgb]{1.00,0.00,0.00}{1,1,1}{\strut ##1}}}}
\@namedef{PY@tok@kc}{\let\PY@bf=\textbf\def\PY@tc##1{\textcolor[rgb]{0.00,0.50,0.00}{##1}}}
\@namedef{PY@tok@kd}{\let\PY@bf=\textbf\def\PY@tc##1{\textcolor[rgb]{0.00,0.50,0.00}{##1}}}
\@namedef{PY@tok@kn}{\let\PY@bf=\textbf\def\PY@tc##1{\textcolor[rgb]{0.00,0.50,0.00}{##1}}}
\@namedef{PY@tok@kr}{\let\PY@bf=\textbf\def\PY@tc##1{\textcolor[rgb]{0.00,0.50,0.00}{##1}}}
\@namedef{PY@tok@bp}{\def\PY@tc##1{\textcolor[rgb]{0.00,0.50,0.00}{##1}}}
\@namedef{PY@tok@fm}{\def\PY@tc##1{\textcolor[rgb]{0.00,0.00,1.00}{##1}}}
\@namedef{PY@tok@vc}{\def\PY@tc##1{\textcolor[rgb]{0.10,0.09,0.49}{##1}}}
\@namedef{PY@tok@vg}{\def\PY@tc##1{\textcolor[rgb]{0.10,0.09,0.49}{##1}}}
\@namedef{PY@tok@vi}{\def\PY@tc##1{\textcolor[rgb]{0.10,0.09,0.49}{##1}}}
\@namedef{PY@tok@vm}{\def\PY@tc##1{\textcolor[rgb]{0.10,0.09,0.49}{##1}}}
\@namedef{PY@tok@sa}{\def\PY@tc##1{\textcolor[rgb]{0.73,0.13,0.13}{##1}}}
\@namedef{PY@tok@sb}{\def\PY@tc##1{\textcolor[rgb]{0.73,0.13,0.13}{##1}}}
\@namedef{PY@tok@sc}{\def\PY@tc##1{\textcolor[rgb]{0.73,0.13,0.13}{##1}}}
\@namedef{PY@tok@dl}{\def\PY@tc##1{\textcolor[rgb]{0.73,0.13,0.13}{##1}}}
\@namedef{PY@tok@s2}{\def\PY@tc##1{\textcolor[rgb]{0.73,0.13,0.13}{##1}}}
\@namedef{PY@tok@sh}{\def\PY@tc##1{\textcolor[rgb]{0.73,0.13,0.13}{##1}}}
\@namedef{PY@tok@s1}{\def\PY@tc##1{\textcolor[rgb]{0.73,0.13,0.13}{##1}}}
\@namedef{PY@tok@mb}{\def\PY@tc##1{\textcolor[rgb]{0.40,0.40,0.40}{##1}}}
\@namedef{PY@tok@mf}{\def\PY@tc##1{\textcolor[rgb]{0.40,0.40,0.40}{##1}}}
\@namedef{PY@tok@mh}{\def\PY@tc##1{\textcolor[rgb]{0.40,0.40,0.40}{##1}}}
\@namedef{PY@tok@mi}{\def\PY@tc##1{\textcolor[rgb]{0.40,0.40,0.40}{##1}}}
\@namedef{PY@tok@il}{\def\PY@tc##1{\textcolor[rgb]{0.40,0.40,0.40}{##1}}}
\@namedef{PY@tok@mo}{\def\PY@tc##1{\textcolor[rgb]{0.40,0.40,0.40}{##1}}}
\@namedef{PY@tok@ch}{\let\PY@it=\textit\def\PY@tc##1{\textcolor[rgb]{0.24,0.48,0.48}{##1}}}
\@namedef{PY@tok@cm}{\let\PY@it=\textit\def\PY@tc##1{\textcolor[rgb]{0.24,0.48,0.48}{##1}}}
\@namedef{PY@tok@cpf}{\let\PY@it=\textit\def\PY@tc##1{\textcolor[rgb]{0.24,0.48,0.48}{##1}}}
\@namedef{PY@tok@c1}{\let\PY@it=\textit\def\PY@tc##1{\textcolor[rgb]{0.24,0.48,0.48}{##1}}}
\@namedef{PY@tok@cs}{\let\PY@it=\textit\def\PY@tc##1{\textcolor[rgb]{0.24,0.48,0.48}{##1}}}

\def\PYZbs{\char`\\}
\def\PYZus{\char`\_}
\def\PYZob{\char`\{}
\def\PYZcb{\char`\}}
\def\PYZca{\char`\^}
\def\PYZam{\char`\&}
\def\PYZlt{\char`\<}
\def\PYZgt{\char`\>}
\def\PYZsh{\char`\#}
\def\PYZpc{\char`\%}
\def\PYZdl{\char`\$}
\def\PYZhy{\char`\-}
\def\PYZsq{\char`\'}
\def\PYZdq{\char`\"}
\def\PYZti{\char`\~}
% for compatibility with earlier versions
\def\PYZat{@}
\def\PYZlb{[}
\def\PYZrb{]}
\makeatother


    % For linebreaks inside Verbatim environment from package fancyvrb.
    \makeatletter
        \newbox\Wrappedcontinuationbox
        \newbox\Wrappedvisiblespacebox
        \newcommand*\Wrappedvisiblespace {\textcolor{red}{\textvisiblespace}}
        \newcommand*\Wrappedcontinuationsymbol {\textcolor{red}{\llap{\tiny$\m@th\hookrightarrow$}}}
        \newcommand*\Wrappedcontinuationindent {3ex }
        \newcommand*\Wrappedafterbreak {\kern\Wrappedcontinuationindent\copy\Wrappedcontinuationbox}
        % Take advantage of the already applied Pygments mark-up to insert
        % potential linebreaks for TeX processing.
        %        {, <, #, %, $, ' and ": go to next line.
        %        _, }, ^, &, >, - and ~: stay at end of broken line.
        % Use of \textquotesingle for straight quote.
        \newcommand*\Wrappedbreaksatspecials {%
            \def\PYGZus{\discretionary{\char`\_}{\Wrappedafterbreak}{\char`\_}}%
            \def\PYGZob{\discretionary{}{\Wrappedafterbreak\char`\{}{\char`\{}}%
            \def\PYGZcb{\discretionary{\char`\}}{\Wrappedafterbreak}{\char`\}}}%
            \def\PYGZca{\discretionary{\char`\^}{\Wrappedafterbreak}{\char`\^}}%
            \def\PYGZam{\discretionary{\char`\&}{\Wrappedafterbreak}{\char`\&}}%
            \def\PYGZlt{\discretionary{}{\Wrappedafterbreak\char`\<}{\char`\<}}%
            \def\PYGZgt{\discretionary{\char`\>}{\Wrappedafterbreak}{\char`\>}}%
            \def\PYGZsh{\discretionary{}{\Wrappedafterbreak\char`\#}{\char`\#}}%
            \def\PYGZpc{\discretionary{}{\Wrappedafterbreak\char`\%}{\char`\%}}%
            \def\PYGZdl{\discretionary{}{\Wrappedafterbreak\char`\$}{\char`\$}}%
            \def\PYGZhy{\discretionary{\char`\-}{\Wrappedafterbreak}{\char`\-}}%
            \def\PYGZsq{\discretionary{}{\Wrappedafterbreak\textquotesingle}{\textquotesingle}}%
            \def\PYGZdq{\discretionary{}{\Wrappedafterbreak\char`\"}{\char`\"}}%
            \def\PYGZti{\discretionary{\char`\~}{\Wrappedafterbreak}{\char`\~}}%
        }
        % Some characters . , ; ? ! / are not pygmentized.
        % This macro makes them "active" and they will insert potential linebreaks
        \newcommand*\Wrappedbreaksatpunct {%
            \lccode`\~`\.\lowercase{\def~}{\discretionary{\hbox{\char`\.}}{\Wrappedafterbreak}{\hbox{\char`\.}}}%
            \lccode`\~`\,\lowercase{\def~}{\discretionary{\hbox{\char`\,}}{\Wrappedafterbreak}{\hbox{\char`\,}}}%
            \lccode`\~`\;\lowercase{\def~}{\discretionary{\hbox{\char`\;}}{\Wrappedafterbreak}{\hbox{\char`\;}}}%
            \lccode`\~`\:\lowercase{\def~}{\discretionary{\hbox{\char`\:}}{\Wrappedafterbreak}{\hbox{\char`\:}}}%
            \lccode`\~`\?\lowercase{\def~}{\discretionary{\hbox{\char`\?}}{\Wrappedafterbreak}{\hbox{\char`\?}}}%
            \lccode`\~`\!\lowercase{\def~}{\discretionary{\hbox{\char`\!}}{\Wrappedafterbreak}{\hbox{\char`\!}}}%
            \lccode`\~`\/\lowercase{\def~}{\discretionary{\hbox{\char`\/}}{\Wrappedafterbreak}{\hbox{\char`\/}}}%
            \catcode`\.\active
            \catcode`\,\active
            \catcode`\;\active
            \catcode`\:\active
            \catcode`\?\active
            \catcode`\!\active
            \catcode`\/\active
            \lccode`\~`\~
        }
    \makeatother

    \let\OriginalVerbatim=\Verbatim
    \makeatletter
    \renewcommand{\Verbatim}[1][1]{%
        %\parskip\z@skip
        \sbox\Wrappedcontinuationbox {\Wrappedcontinuationsymbol}%
        \sbox\Wrappedvisiblespacebox {\FV@SetupFont\Wrappedvisiblespace}%
        \def\FancyVerbFormatLine ##1{\hsize\linewidth
            \vtop{\raggedright\hyphenpenalty\z@\exhyphenpenalty\z@
                \doublehyphendemerits\z@\finalhyphendemerits\z@
                \strut ##1\strut}%
        }%
        % If the linebreak is at a space, the latter will be displayed as visible
        % space at end of first line, and a continuation symbol starts next line.
        % Stretch/shrink are however usually zero for typewriter font.
        \def\FV@Space {%
            \nobreak\hskip\z@ plus\fontdimen3\font minus\fontdimen4\font
            \discretionary{\copy\Wrappedvisiblespacebox}{\Wrappedafterbreak}
            {\kern\fontdimen2\font}%
        }%

        % Allow breaks at special characters using \PYG... macros.
        \Wrappedbreaksatspecials
        % Breaks at punctuation characters . , ; ? ! and / need catcode=\active
        \OriginalVerbatim[#1,codes*=\Wrappedbreaksatpunct]%
    }
    \makeatother

    % Exact colors from NB
    \definecolor{incolor}{HTML}{303F9F}
    \definecolor{outcolor}{HTML}{D84315}
    \definecolor{cellborder}{HTML}{CFCFCF}
    \definecolor{cellbackground}{HTML}{F7F7F7}

    % prompt
    \makeatletter
    \newcommand{\boxspacing}{\kern\kvtcb@left@rule\kern\kvtcb@boxsep}
    \makeatother
    \newcommand{\prompt}[4]{
        {\ttfamily\llap{{\color{#2}[#3]:\hspace{3pt}#4}}\vspace{-\baselineskip}}
    }
    

    
    % Prevent overflowing lines due to hard-to-break entities
    \sloppy
    % Setup hyperref package
    \hypersetup{
      breaklinks=true,  % so long urls are correctly broken across lines
      colorlinks=true,
      urlcolor=urlcolor,
      linkcolor=linkcolor,
      citecolor=citecolor,
      }
    % Slightly bigger margins than the latex defaults
    
    \geometry{verbose,tmargin=1in,bmargin=1in,lmargin=1in,rmargin=1in}
    
    

\begin{document}
    
    \maketitle
    
    

    
    

    \section{Homework 7 (Due 28 Mar)}\label{homework-7-due-28-mar}

\textbf{Due 28 Mar (midnight)}

Total points: 100

    \begin{tcolorbox}[breakable, size=fbox, boxrule=1pt, pad at break*=1mm,colback=cellbackground, colframe=cellborder]
\prompt{In}{incolor}{1}{\boxspacing}
\begin{Verbatim}[commandchars=\\\{\}]
\PY{k+kn}{import}\PY{+w}{ }\PY{n+nn}{numpy}\PY{+w}{ }\PY{k}{as}\PY{+w}{ }\PY{n+nn}{np}
\PY{k+kn}{from}\PY{+w}{ }\PY{n+nn}{math}\PY{+w}{ }\PY{k+kn}{import} \PY{o}{*}
\PY{k+kn}{import}\PY{+w}{ }\PY{n+nn}{matplotlib}\PY{n+nn}{.}\PY{n+nn}{pyplot}\PY{+w}{ }\PY{k}{as}\PY{+w}{ }\PY{n+nn}{plt}
\PY{k+kn}{import}\PY{+w}{ }\PY{n+nn}{pandas}\PY{+w}{ }\PY{k}{as}\PY{+w}{ }\PY{n+nn}{pd}
\PY{o}{\PYZpc{}}\PY{k}{matplotlib} inline
\PY{n}{plt}\PY{o}{.}\PY{n}{style}\PY{o}{.}\PY{n}{use}\PY{p}{(}\PY{l+s+s1}{\PYZsq{}}\PY{l+s+s1}{seaborn\PYZhy{}v0\PYZus{}8\PYZhy{}colorblind}\PY{l+s+s1}{\PYZsq{}}\PY{p}{)}
\end{Verbatim}
\end{tcolorbox}

    \subsubsection{Exercise 1 (10pt) Where does the energy
go?}\label{exercise-1-10pt-where-does-the-energy-go}

The damped harmonic oscillator is described by the equation of motion
is:

\[m\ddot{x} + b\dot{x} + kx = 0\]

where \(m\) is the mass, \(b\) is the damping coefficient, and \(k\) is
the spring constant.

The damping term (\(F_{damp} = - b\dot{x}\)) models the dissipative
forces acting on the oscillator. The total energy for the oscillator is
given by the sum of the kinetic and potential energies,

\[E = \frac{1}{2}m\dot{x}^2 + \frac{1}{2}kx^2.\]

\begin{itemize}
\tightlist
\item
  1a (3pt). What is the energy per unit time dissipated by the damping
  force?
\item
  1b (4pt). Take the time derivative of the total energy and show that
  it is equal (in magnitude) to the energy dissipated by the damping
  force.
\item
  1c (3pt). What is the sign relationship between the energy dissipated
  by the damping force and the time derivative of the total energy?
\end{itemize}

    \subsubsection{Exercise 2 (10pt), Unpacking the critically damped
solution}\label{exercise-2-10pt-unpacking-the-critically-damped-solution}

The solution for critical damping (\(\beta = \omega_0\)) is given by,

\[x(t) = x_1(t) + x_2(t) = Ae^{-\beta t} + Bte^{-\beta t},\]

where \(A\) and \(B\) are constants. Notice the second solution
\(x_2(t) = Bte^{-\omega_0t}\) has an additional linear term \(t\). We
glossed over this solution in class, but it is important to understand
why this term is present because it tells us about solving differential
equations with pathologically difficult-to-see solutions.

Start with the under damped solutions,

\[y_1(t) = e^{-\beta t}\cos(\omega_1 t) \quad \text{and} \quad y_2(t) = e^{-\beta t}\sin(\omega_1 t),\]

where we have used the notation
\(\omega_1 = \sqrt{\omega_0^2 - \beta^2}\).

\begin{itemize}
\tightlist
\item
  2a (3pt). Show that you can recover the first solution \(x_1(t)\) by
  taking the limit of \(\beta \rightarrow \omega_0\) of \(y_1(t)\).
\item
  2b (3pt). Show that you cannot recover the second solution \(x_2(t)\)
  by taking the limit of \(\beta \rightarrow \omega_0\) of \(y_2(t)\)
  directly. What do you get?
\item
  2c (4pt). If \(\beta \neq \omega_0\), you can divide \(y_2(t)\) by
  \(\omega_1\). Now show that in the limit
  \(\beta \rightarrow \omega_0\) of \(y_2(t)/\omega_1\), we recover the
  form of \(x_2(t)\).
\end{itemize}

    \subsubsection{Exercise 3 (20pt), Exploring the damped harmonic
oscillator}\label{exercise-3-20pt-exploring-the-damped-harmonic-oscillator}

You can choose to solve this exercise using analytical, or graphical
methods. Should you decide to use numerical methods, you should make
sure to use a method that conserves energy. However, this exercise does
not need to be solved numerically as closed form solutions exist.

The solution to the simple harmonic oscillator is given by,

\[x(t) = A_{sho}\cos(\omega_0 t) + B_{sho}\sin(\omega_0 t),\]

where \(A_{sho}\) and \(B_{sho}\) are determined by the initial
conditions. The solution to the underdamped harmonic oscillator is given
by,

\[x(t) = e^{-\beta t}\left(A\cos(\omega_1 t) + B\sin(\omega_1 t)\right),\]

where \(A\) and \(B\) are also determined by the initial conditions.

\paragraph{Scenario 1}\label{scenario-1}

An undamped oscillator has a period \(T_0 = 1.000\mathrm{s}\), but then
the damping is turned on. We observe the period increases by 0.001
seconds.

\begin{itemize}
\tightlist
\item
  3a (3pt). What is the damping coefficient \(\beta\)?
\item
  3b (3pt). By how much does the amplitude of the oscillation decrease
  after 10 periods?
\item
  3c (3pt). What would be able to notice more easily, the change in
  period or the change in amplitude?
\end{itemize}

\paragraph{Scenario 2}\label{scenario-2}

An undamped oscillator has a period \(T_0 = 1.000\mathrm{s}\). With weak
damping, the amplitude drops by 50\% in one period \(T_1\). Recall the
period of the damped oscillator is given by \(T_1 = 2\pi/\omega_1\).

\begin{itemize}
\item
  3d (3pt). What is the damping coefficient \(\beta\)?
\item
  3e (3pt). What is the period of the damped oscillator?
\item
  3f (3pt). By how much does the amplitude of the oscillation decrease
  after 10 periods?
\item
  3g (2pt). What would be able to notice more easily, the change in
  period or the change in amplitude?
\end{itemize}

    \subsubsection{Exercise 4 (40pt), The Damped Driven
Oscillator}\label{exercise-4-40pt-the-damped-driven-oscillator}

The damped driven oscillator is described by the equation of motion,

\[m\ddot{x} + b\dot{x} + kx = F(t),\]

where \(F_0\) is the amplitude of the driving force and \(\omega\) is
the frequency of the driving force. We reduced this equation by dividing
by \(m\) to get the equation of motion in terms of the damping
coefficient \(\beta = b/2m\) and the natural frequency
\(\omega_0 = \sqrt{k/m}\). In the equation below, \(f(t) = F_0/m\).

\[\ddot{x} + 2\beta\dot{x} + \omega_0^2x = f(t).\]

We solve this equation for sinusoidal driving forces,
\(f(t) = f_0\cos(\omega t)\) and demonstrated the resonance effect. In
this exercise, we will numerically solve the equation of motion. This
will allow us to explore the behavior of the damped driven oscillator
for different periodic driving forces, not just sinusoidal ones.

\begin{itemize}
\tightlist
\item
  4a (10pt). Modify the code (or write your own) include a driving force
  \(f(t) = f_0\cos(\omega t)\).

  \begin{itemize}
  \tightlist
  \item
    To start, \textbf{choose \(f_0\)=1, \(\omega_0\)=10 and
    \(\beta\)=0.1.}
  \item
    Check that your code produces a steady state solution for the driven
    harmonic oscillator. Roughly, what is the steady state amplitude,
    \(A(t \rightarrow \infty)\), of the driven oscillator?
  \end{itemize}
\item
  4b (10pt). For different choices of the driving frequency \(\omega\),
  observe the behavior of the driven oscillator. Describe the behavior
  of the driven oscillator for \(\omega \ll \omega_0\),
  \(\omega \approx \omega_0\), and \(\omega \gg \omega_0\).
\item
  4c (20 pt). Sweep the driving frequency \(\omega\) at a fixed
  amplitude and store the amplitude of the steady-state solution. Plot
  the amplitude of the steady-state solution as a function of the
  driving frequency \(\omega\). Describe the behavior of the amplitude
  as a function of the driving frequency.
\end{itemize}

We have written a numerical solver for the damped undriven oscillator;
it's similar to the prior codes we've used. Your task is to modify the
code to include a driving force. We have used the second order
Runge-Kutta method to solve the equation of motion. You can use the code
as a starting point for the damped driven oscillator. Notice we also
call the code in the next cell.

    \begin{tcolorbox}[breakable, size=fbox, boxrule=1pt, pad at break*=1mm,colback=cellbackground, colframe=cellborder]
\prompt{In}{incolor}{9}{\boxspacing}
\begin{Verbatim}[commandchars=\\\{\}]
\PY{k}{def}\PY{+w}{ }\PY{n+nf}{ddho}\PY{p}{(}\PY{n}{y}\PY{p}{,} \PY{n}{t}\PY{p}{,} \PY{n}{omega\PYZus{}0}\PY{p}{,} \PY{n}{beta}\PY{p}{)}\PY{p}{:}
\PY{+w}{    }\PY{l+s+sd}{\PYZdq{}\PYZdq{}\PYZdq{}}
\PY{l+s+sd}{    Function to compute derivatives based on the damped driven oscillator model.}
\PY{l+s+sd}{    Notice the driving force is included but zero.}
\PY{l+s+sd}{    \PYZdq{}\PYZdq{}\PYZdq{}}
    \PY{n}{x}\PY{p}{,} \PY{n}{v} \PY{o}{=} \PY{n}{y}
    \PY{n}{dxdt} \PY{o}{=} \PY{n}{v}
    \PY{n}{dvdt} \PY{o}{=} \PY{o}{\PYZhy{}}\PY{l+m+mi}{2} \PY{o}{*} \PY{n}{beta} \PY{o}{*} \PY{n}{v} \PY{o}{\PYZhy{}} \PY{n}{omega\PYZus{}0}\PY{o}{*}\PY{o}{*}\PY{l+m+mi}{2} \PY{o}{*} \PY{n}{x}
    \PY{k}{return} \PY{n}{np}\PY{o}{.}\PY{n}{array}\PY{p}{(}\PY{p}{[}\PY{n}{dxdt}\PY{p}{,} \PY{n}{dvdt}\PY{p}{]}\PY{p}{)}


\PY{k}{def}\PY{+w}{ }\PY{n+nf}{rk2\PYZus{}step}\PY{p}{(}\PY{n}{y}\PY{p}{,} \PY{n}{t}\PY{p}{,} \PY{n}{dt}\PY{p}{,} \PY{n}{omega\PYZus{}0}\PY{p}{,} \PY{n}{beta}\PY{p}{,} \PY{n}{F0}\PY{o}{=}\PY{l+m+mf}{1.0}\PY{p}{,} \PY{n}{omega}\PY{o}{=}\PY{l+m+mi}{10}\PY{p}{,} \PY{n}{phase}\PY{o}{=}\PY{l+m+mf}{0.0}\PY{p}{)}\PY{p}{:}
\PY{+w}{    }\PY{l+s+sd}{\PYZdq{}\PYZdq{}\PYZdq{}}
\PY{l+s+sd}{    Performs a single step of the Runge\PYZhy{}Kutta 2nd order (midpoint) method.}

\PY{l+s+sd}{    Parameters:}
\PY{l+s+sd}{    \PYZhy{} y: current state [position, velocity].}
\PY{l+s+sd}{    \PYZhy{} t: current time.}
\PY{l+s+sd}{    \PYZhy{} dt: time step size.}
\PY{l+s+sd}{    \PYZhy{} omega\PYZus{}0: natural frequency of the oscillator.}
\PY{l+s+sd}{    \PYZhy{} beta: damping coefficient.}
\PY{l+s+sd}{    \PYZhy{} F0: amplitude of the driving force.}
\PY{l+s+sd}{    \PYZhy{} omega: frequency of the driving force.}
\PY{l+s+sd}{    \PYZhy{} phase: phase of the driving force.}

\PY{l+s+sd}{    Returns:}
\PY{l+s+sd}{    \PYZhy{} y\PYZus{}next: state at time step dt [position, velocity].}
\PY{l+s+sd}{    \PYZdq{}\PYZdq{}\PYZdq{}}
    \PY{n}{k1} \PY{o}{=} \PY{n}{ddho}\PY{p}{(}\PY{n}{y}\PY{p}{,} \PY{n}{t}\PY{p}{,} \PY{n}{omega\PYZus{}0}\PY{p}{,} \PY{n}{beta}\PY{p}{)}
    \PY{n}{k2} \PY{o}{=} \PY{n}{ddho}\PY{p}{(}\PY{n}{y} \PY{o}{+} \PY{l+m+mf}{0.5} \PY{o}{*} \PY{n}{k1} \PY{o}{*} \PY{n}{dt}\PY{p}{,} \PY{n}{t} \PY{o}{+} \PY{l+m+mf}{0.5} \PY{o}{*} \PY{n}{dt}\PY{p}{,} \PY{n}{omega\PYZus{}0}\PY{p}{,} \PY{n}{beta}\PY{p}{)}
    \PY{n}{y\PYZus{}next} \PY{o}{=} \PY{n}{y} \PY{o}{+} \PY{n}{k2} \PY{o}{*} \PY{n}{dt}
    
    \PY{k}{return} \PY{n}{y\PYZus{}next}
\end{Verbatim}
\end{tcolorbox}

    \begin{tcolorbox}[breakable, size=fbox, boxrule=1pt, pad at break*=1mm,colback=cellbackground, colframe=cellborder]
\prompt{In}{incolor}{10}{\boxspacing}
\begin{Verbatim}[commandchars=\\\{\}]
\PY{c+c1}{\PYZsh{} Simulation parameters}
\PY{n}{T} \PY{o}{=} \PY{l+m+mf}{100.0}  \PY{c+c1}{\PYZsh{} total time}
\PY{n}{dt} \PY{o}{=} \PY{l+m+mf}{0.01}  \PY{c+c1}{\PYZsh{} time step}
\PY{n}{omega\PYZus{}0} \PY{o}{=} \PY{l+m+mi}{10}  \PY{c+c1}{\PYZsh{} natural frequency}
\PY{n}{beta} \PY{o}{=} \PY{l+m+mf}{0.1}  \PY{c+c1}{\PYZsh{} damping coefficient}

\PY{n}{steps} \PY{o}{=} \PY{n+nb}{int}\PY{p}{(}\PY{n}{T} \PY{o}{/} \PY{n}{dt}\PY{p}{)}
\PY{n}{t} \PY{o}{=} \PY{n}{np}\PY{o}{.}\PY{n}{linspace}\PY{p}{(}\PY{l+m+mi}{0}\PY{p}{,} \PY{n}{T}\PY{p}{,} \PY{n}{steps}\PY{p}{)}
\PY{n}{y} \PY{o}{=} \PY{n}{np}\PY{o}{.}\PY{n}{zeros}\PY{p}{(}\PY{p}{(}\PY{n}{steps}\PY{p}{,} \PY{l+m+mi}{2}\PY{p}{)}\PY{p}{)}  \PY{c+c1}{\PYZsh{} Array to hold position and velocity}

\PY{c+c1}{\PYZsh{}\PYZsh{} Initial conditions}
\PY{n}{y}\PY{p}{[}\PY{l+m+mi}{0}\PY{p}{]} \PY{o}{=} \PY{p}{[}\PY{l+m+mf}{0.25}\PY{p}{,} \PY{l+m+mf}{0.0}\PY{p}{]}  \PY{c+c1}{\PYZsh{} x=1, v=0}

\PY{k}{for} \PY{n}{i} \PY{o+ow}{in} \PY{n+nb}{range}\PY{p}{(}\PY{n}{steps} \PY{o}{\PYZhy{}} \PY{l+m+mi}{1}\PY{p}{)}\PY{p}{:}
    \PY{n}{y}\PY{p}{[}\PY{n}{i} \PY{o}{+} \PY{l+m+mi}{1}\PY{p}{]} \PY{o}{=} \PY{n}{rk2\PYZus{}step}\PY{p}{(}\PY{n}{y}\PY{p}{[}\PY{n}{i}\PY{p}{]}\PY{p}{,} \PY{n}{t}\PY{p}{[}\PY{n}{i}\PY{p}{]}\PY{p}{,} \PY{n}{dt}\PY{p}{,} \PY{n}{omega\PYZus{}0}\PY{p}{,} \PY{n}{beta}\PY{p}{)}

\PY{n}{oscillator} \PY{o}{=} \PY{n}{pd}\PY{o}{.}\PY{n}{DataFrame}\PY{p}{(}\PY{n}{y}\PY{p}{,} \PY{n}{columns}\PY{o}{=}\PY{p}{[}\PY{l+s+s2}{\PYZdq{}}\PY{l+s+s2}{Position}\PY{l+s+s2}{\PYZdq{}}\PY{p}{,} \PY{l+s+s2}{\PYZdq{}}\PY{l+s+s2}{Velocity}\PY{l+s+s2}{\PYZdq{}}\PY{p}{]}\PY{p}{)}
\PY{n}{oscillator}\PY{p}{[}\PY{l+s+s2}{\PYZdq{}}\PY{l+s+s2}{Time}\PY{l+s+s2}{\PYZdq{}}\PY{p}{]} \PY{o}{=} \PY{n}{t}

\PY{c+c1}{\PYZsh{} Plotting the results}
\PY{n}{plt}\PY{o}{.}\PY{n}{figure}\PY{p}{(}\PY{n}{figsize}\PY{o}{=}\PY{p}{(}\PY{l+m+mi}{10}\PY{p}{,} \PY{l+m+mi}{8}\PY{p}{)}\PY{p}{)}
\PY{n}{plt}\PY{o}{.}\PY{n}{plot}\PY{p}{(}\PY{n}{oscillator}\PY{p}{[}\PY{l+s+s2}{\PYZdq{}}\PY{l+s+s2}{Time}\PY{l+s+s2}{\PYZdq{}}\PY{p}{]}\PY{p}{,} \PY{n}{oscillator}\PY{p}{[}\PY{l+s+s2}{\PYZdq{}}\PY{l+s+s2}{Position}\PY{l+s+s2}{\PYZdq{}}\PY{p}{]}\PY{p}{,} \PY{n}{label}\PY{o}{=}\PY{l+s+s2}{\PYZdq{}}\PY{l+s+s2}{Damped Oscillator}\PY{l+s+s2}{\PYZdq{}}\PY{p}{)}
\PY{n}{plt}\PY{o}{.}\PY{n}{axhline}\PY{p}{(}\PY{l+m+mi}{0}\PY{p}{,} \PY{n}{color}\PY{o}{=}\PY{l+s+s2}{\PYZdq{}}\PY{l+s+s2}{black}\PY{l+s+s2}{\PYZdq{}}\PY{p}{,} \PY{n}{linewidth}\PY{o}{=}\PY{l+m+mi}{1}\PY{p}{)}
\PY{n}{plt}\PY{o}{.}\PY{n}{xlabel}\PY{p}{(}\PY{l+s+s2}{\PYZdq{}}\PY{l+s+s2}{Time}\PY{l+s+s2}{\PYZdq{}}\PY{p}{)}
\PY{n}{plt}\PY{o}{.}\PY{n}{ylabel}\PY{p}{(}\PY{l+s+s2}{\PYZdq{}}\PY{l+s+s2}{Position}\PY{l+s+s2}{\PYZdq{}}\PY{p}{)}
\PY{n}{plt}\PY{o}{.}\PY{n}{grid}\PY{p}{(}\PY{k+kc}{True}\PY{p}{)}
\PY{n}{plt}\PY{o}{.}\PY{n}{legend}\PY{p}{(}\PY{p}{)}
\end{Verbatim}
\end{tcolorbox}

            \begin{tcolorbox}[breakable, size=fbox, boxrule=.5pt, pad at break*=1mm, opacityfill=0]
\prompt{Out}{outcolor}{10}{\boxspacing}
\begin{Verbatim}[commandchars=\\\{\}]
<matplotlib.legend.Legend at 0x11bfac2d0>
\end{Verbatim}
\end{tcolorbox}
        
    \begin{center}
    \adjustimage{max size={0.9\linewidth}{0.9\paperheight}}{hw7_files/hw7_8_1.png}
    \end{center}
    { \hspace*{\fill} \\}
    
    \subsubsection{Exercise 5 (20pt), Project Check-in
1}\label{exercise-5-20pt-project-check-in-1}

This is the first of three check-ins for your final project. Last week,
you submitted a project proposal with milestones. This week, you will
submit a progress report. This is common in science and engineering
research. Funding agencies require an annual report to ensure the
project is on track. While we are not funding you, we want this work to
be a similar experience to a research project.

The first annual report is often a minimal progress report. How much can
you say? You have only just started!

But, you can say what you have done, what you are planning to do, what
you have learned so far, and what changes you have to make. This is the
purpose of this exercise.

\begin{itemize}
\tightlist
\item
  5a (5 pts). Review your project proposal. What have you been able to
  accomplish so far? What were you unable to do in the time you had? Be
  honest in your evaluation of your progress. You will not be penalized
  for not reaching your milestones. What does that mean you need to
  prioritize in the coming weeks? (at least 250 words)
\item
  5b (5 pts). What problems did you encounter in doing you research?
  What questions came up and how did you resolve them? Are there any
  unresolved questions? (at least 250 words)
\item
  5c (5 pts). Provide an artifact from your project. This could be a
  plot, a code snippet, a data set, or a figure. Explain what this
  artifact is and how it fits into your project. (at least 100-200
  words)
\item
  5d (5 pts). Update your project timeline and milestones. How will you
  adjust your timeline to account for the work you have done and the
  work you have left to do? (at least 100-200 words)
\end{itemize}

    \subsubsection{Extra Credit - Integrating Classwork With
Research}\label{extra-credit---integrating-classwork-with-research}

This opportunity will allow you to earn up to 5 extra credit points on a
Homework per week. These points can push you above 100\% or help make up
for missed exercises. In order to earn all points you must:

\begin{enumerate}
\def\labelenumi{\arabic{enumi}.}
\item
  Attend an MSU research talk (recommended research oriented Clubs is
  provided below)
\item
  Summarize the talk using at least 150 words
\item
  Turn in the summary along with your Homework.
\end{enumerate}

Approved talks: Talks given by researchers through the following clubs:
* Research and Idea Sharing Enterprise (RAISE)\hspace{0pt}: Meets
Wednesday Nights Society for Physics Students (SPS)\hspace{0pt}: Meets
Monday Nights

\begin{itemize}
\item
  Astronomy Club\hspace{0pt}: Meets Monday Nights
\item
  Facility For Rare Isotope Beam (FRIB) Seminars: \hspace{0pt}Occur
  multiple times a week
\end{itemize}

    


    % Add a bibliography block to the postdoc
    
    
    
\end{document}
