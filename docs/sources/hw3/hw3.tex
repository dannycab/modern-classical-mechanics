\documentclass[11pt]{article}

    \usepackage[breakable]{tcolorbox}
    \usepackage{parskip} % Stop auto-indenting (to mimic markdown behaviour)
    

    % Basic figure setup, for now with no caption control since it's done
    % automatically by Pandoc (which extracts ![](path) syntax from Markdown).
    \usepackage{graphicx}
    % Keep aspect ratio if custom image width or height is specified
    \setkeys{Gin}{keepaspectratio}
    % Maintain compatibility with old templates. Remove in nbconvert 6.0
    \let\Oldincludegraphics\includegraphics
    % Ensure that by default, figures have no caption (until we provide a
    % proper Figure object with a Caption API and a way to capture that
    % in the conversion process - todo).
    \usepackage{caption}
    \DeclareCaptionFormat{nocaption}{}
    \captionsetup{format=nocaption,aboveskip=0pt,belowskip=0pt}

    \usepackage{float}
    \floatplacement{figure}{H} % forces figures to be placed at the correct location
    \usepackage{xcolor} % Allow colors to be defined
    \usepackage{enumerate} % Needed for markdown enumerations to work
    \usepackage{geometry} % Used to adjust the document margins
    \usepackage{amsmath} % Equations
    \usepackage{amssymb} % Equations
    \usepackage{textcomp} % defines textquotesingle
    % Hack from http://tex.stackexchange.com/a/47451/13684:
    \AtBeginDocument{%
        \def\PYZsq{\textquotesingle}% Upright quotes in Pygmentized code
    }
    \usepackage{upquote} % Upright quotes for verbatim code
    \usepackage{eurosym} % defines \euro

    \usepackage{iftex}
    \ifPDFTeX
        \usepackage[T1]{fontenc}
        \IfFileExists{alphabeta.sty}{
              \usepackage{alphabeta}
          }{
              \usepackage[mathletters]{ucs}
              \usepackage[utf8x]{inputenc}
          }
    \else
        \usepackage{fontspec}
        \usepackage{unicode-math}
    \fi

    \usepackage{fancyvrb} % verbatim replacement that allows latex
    \usepackage{grffile} % extends the file name processing of package graphics
                         % to support a larger range
    \makeatletter % fix for old versions of grffile with XeLaTeX
    \@ifpackagelater{grffile}{2019/11/01}
    {
      % Do nothing on new versions
    }
    {
      \def\Gread@@xetex#1{%
        \IfFileExists{"\Gin@base".bb}%
        {\Gread@eps{\Gin@base.bb}}%
        {\Gread@@xetex@aux#1}%
      }
    }
    \makeatother
    \usepackage[Export]{adjustbox} % Used to constrain images to a maximum size
    \adjustboxset{max size={0.9\linewidth}{0.9\paperheight}}

    % The hyperref package gives us a pdf with properly built
    % internal navigation ('pdf bookmarks' for the table of contents,
    % internal cross-reference links, web links for URLs, etc.)
    \usepackage{hyperref}
    % The default LaTeX title has an obnoxious amount of whitespace. By default,
    % titling removes some of it. It also provides customization options.
    \usepackage{titling}
    \usepackage{longtable} % longtable support required by pandoc >1.10
    \usepackage{booktabs}  % table support for pandoc > 1.12.2
    \usepackage{array}     % table support for pandoc >= 2.11.3
    \usepackage{calc}      % table minipage width calculation for pandoc >= 2.11.1
    \usepackage[inline]{enumitem} % IRkernel/repr support (it uses the enumerate* environment)
    \usepackage[normalem]{ulem} % ulem is needed to support strikethroughs (\sout)
                                % normalem makes italics be italics, not underlines
    \usepackage{soul}      % strikethrough (\st) support for pandoc >= 3.0.0
    \usepackage{mathrsfs}
    

    
    % Colors for the hyperref package
    \definecolor{urlcolor}{rgb}{0,.145,.698}
    \definecolor{linkcolor}{rgb}{.71,0.21,0.01}
    \definecolor{citecolor}{rgb}{.12,.54,.11}

    % ANSI colors
    \definecolor{ansi-black}{HTML}{3E424D}
    \definecolor{ansi-black-intense}{HTML}{282C36}
    \definecolor{ansi-red}{HTML}{E75C58}
    \definecolor{ansi-red-intense}{HTML}{B22B31}
    \definecolor{ansi-green}{HTML}{00A250}
    \definecolor{ansi-green-intense}{HTML}{007427}
    \definecolor{ansi-yellow}{HTML}{DDB62B}
    \definecolor{ansi-yellow-intense}{HTML}{B27D12}
    \definecolor{ansi-blue}{HTML}{208FFB}
    \definecolor{ansi-blue-intense}{HTML}{0065CA}
    \definecolor{ansi-magenta}{HTML}{D160C4}
    \definecolor{ansi-magenta-intense}{HTML}{A03196}
    \definecolor{ansi-cyan}{HTML}{60C6C8}
    \definecolor{ansi-cyan-intense}{HTML}{258F8F}
    \definecolor{ansi-white}{HTML}{C5C1B4}
    \definecolor{ansi-white-intense}{HTML}{A1A6B2}
    \definecolor{ansi-default-inverse-fg}{HTML}{FFFFFF}
    \definecolor{ansi-default-inverse-bg}{HTML}{000000}

    % common color for the border for error outputs.
    \definecolor{outerrorbackground}{HTML}{FFDFDF}

    % commands and environments needed by pandoc snippets
    % extracted from the output of `pandoc -s`
    \providecommand{\tightlist}{%
      \setlength{\itemsep}{0pt}\setlength{\parskip}{0pt}}
    \DefineVerbatimEnvironment{Highlighting}{Verbatim}{commandchars=\\\{\}}
    % Add ',fontsize=\small' for more characters per line
    \newenvironment{Shaded}{}{}
    \newcommand{\KeywordTok}[1]{\textcolor[rgb]{0.00,0.44,0.13}{\textbf{{#1}}}}
    \newcommand{\DataTypeTok}[1]{\textcolor[rgb]{0.56,0.13,0.00}{{#1}}}
    \newcommand{\DecValTok}[1]{\textcolor[rgb]{0.25,0.63,0.44}{{#1}}}
    \newcommand{\BaseNTok}[1]{\textcolor[rgb]{0.25,0.63,0.44}{{#1}}}
    \newcommand{\FloatTok}[1]{\textcolor[rgb]{0.25,0.63,0.44}{{#1}}}
    \newcommand{\CharTok}[1]{\textcolor[rgb]{0.25,0.44,0.63}{{#1}}}
    \newcommand{\StringTok}[1]{\textcolor[rgb]{0.25,0.44,0.63}{{#1}}}
    \newcommand{\CommentTok}[1]{\textcolor[rgb]{0.38,0.63,0.69}{\textit{{#1}}}}
    \newcommand{\OtherTok}[1]{\textcolor[rgb]{0.00,0.44,0.13}{{#1}}}
    \newcommand{\AlertTok}[1]{\textcolor[rgb]{1.00,0.00,0.00}{\textbf{{#1}}}}
    \newcommand{\FunctionTok}[1]{\textcolor[rgb]{0.02,0.16,0.49}{{#1}}}
    \newcommand{\RegionMarkerTok}[1]{{#1}}
    \newcommand{\ErrorTok}[1]{\textcolor[rgb]{1.00,0.00,0.00}{\textbf{{#1}}}}
    \newcommand{\NormalTok}[1]{{#1}}

    % Additional commands for more recent versions of Pandoc
    \newcommand{\ConstantTok}[1]{\textcolor[rgb]{0.53,0.00,0.00}{{#1}}}
    \newcommand{\SpecialCharTok}[1]{\textcolor[rgb]{0.25,0.44,0.63}{{#1}}}
    \newcommand{\VerbatimStringTok}[1]{\textcolor[rgb]{0.25,0.44,0.63}{{#1}}}
    \newcommand{\SpecialStringTok}[1]{\textcolor[rgb]{0.73,0.40,0.53}{{#1}}}
    \newcommand{\ImportTok}[1]{{#1}}
    \newcommand{\DocumentationTok}[1]{\textcolor[rgb]{0.73,0.13,0.13}{\textit{{#1}}}}
    \newcommand{\AnnotationTok}[1]{\textcolor[rgb]{0.38,0.63,0.69}{\textbf{\textit{{#1}}}}}
    \newcommand{\CommentVarTok}[1]{\textcolor[rgb]{0.38,0.63,0.69}{\textbf{\textit{{#1}}}}}
    \newcommand{\VariableTok}[1]{\textcolor[rgb]{0.10,0.09,0.49}{{#1}}}
    \newcommand{\ControlFlowTok}[1]{\textcolor[rgb]{0.00,0.44,0.13}{\textbf{{#1}}}}
    \newcommand{\OperatorTok}[1]{\textcolor[rgb]{0.40,0.40,0.40}{{#1}}}
    \newcommand{\BuiltInTok}[1]{{#1}}
    \newcommand{\ExtensionTok}[1]{{#1}}
    \newcommand{\PreprocessorTok}[1]{\textcolor[rgb]{0.74,0.48,0.00}{{#1}}}
    \newcommand{\AttributeTok}[1]{\textcolor[rgb]{0.49,0.56,0.16}{{#1}}}
    \newcommand{\InformationTok}[1]{\textcolor[rgb]{0.38,0.63,0.69}{\textbf{\textit{{#1}}}}}
    \newcommand{\WarningTok}[1]{\textcolor[rgb]{0.38,0.63,0.69}{\textbf{\textit{{#1}}}}}
    \makeatletter
    \newsavebox\pandoc@box
    \newcommand*\pandocbounded[1]{%
      \sbox\pandoc@box{#1}%
      % scaling factors for width and height
      \Gscale@div\@tempa\textheight{\dimexpr\ht\pandoc@box+\dp\pandoc@box\relax}%
      \Gscale@div\@tempb\linewidth{\wd\pandoc@box}%
      % select the smaller of both
      \ifdim\@tempb\p@<\@tempa\p@
        \let\@tempa\@tempb
      \fi
      % scaling accordingly (\@tempa < 1)
      \ifdim\@tempa\p@<\p@
        \scalebox{\@tempa}{\usebox\pandoc@box}%
      % scaling not needed, use as it is
      \else
        \usebox{\pandoc@box}%
      \fi
    }
    \makeatother

    % Define a nice break command that doesn't care if a line doesn't already
    % exist.
    \def\br{\hspace*{\fill} \\* }
    % Math Jax compatibility definitions
    \def\gt{>}
    \def\lt{<}
    \let\Oldtex\TeX
    \let\Oldlatex\LaTeX
    \renewcommand{\TeX}{\textrm{\Oldtex}}
    \renewcommand{\LaTeX}{\textrm{\Oldlatex}}
    % Document parameters
    % Document title
    \title{hw3}
    
    
    
    
    
    
    
% Pygments definitions
\makeatletter
\def\PY@reset{\let\PY@it=\relax \let\PY@bf=\relax%
    \let\PY@ul=\relax \let\PY@tc=\relax%
    \let\PY@bc=\relax \let\PY@ff=\relax}
\def\PY@tok#1{\csname PY@tok@#1\endcsname}
\def\PY@toks#1+{\ifx\relax#1\empty\else%
    \PY@tok{#1}\expandafter\PY@toks\fi}
\def\PY@do#1{\PY@bc{\PY@tc{\PY@ul{%
    \PY@it{\PY@bf{\PY@ff{#1}}}}}}}
\def\PY#1#2{\PY@reset\PY@toks#1+\relax+\PY@do{#2}}

\@namedef{PY@tok@w}{\def\PY@tc##1{\textcolor[rgb]{0.73,0.73,0.73}{##1}}}
\@namedef{PY@tok@c}{\let\PY@it=\textit\def\PY@tc##1{\textcolor[rgb]{0.24,0.48,0.48}{##1}}}
\@namedef{PY@tok@cp}{\def\PY@tc##1{\textcolor[rgb]{0.61,0.40,0.00}{##1}}}
\@namedef{PY@tok@k}{\let\PY@bf=\textbf\def\PY@tc##1{\textcolor[rgb]{0.00,0.50,0.00}{##1}}}
\@namedef{PY@tok@kp}{\def\PY@tc##1{\textcolor[rgb]{0.00,0.50,0.00}{##1}}}
\@namedef{PY@tok@kt}{\def\PY@tc##1{\textcolor[rgb]{0.69,0.00,0.25}{##1}}}
\@namedef{PY@tok@o}{\def\PY@tc##1{\textcolor[rgb]{0.40,0.40,0.40}{##1}}}
\@namedef{PY@tok@ow}{\let\PY@bf=\textbf\def\PY@tc##1{\textcolor[rgb]{0.67,0.13,1.00}{##1}}}
\@namedef{PY@tok@nb}{\def\PY@tc##1{\textcolor[rgb]{0.00,0.50,0.00}{##1}}}
\@namedef{PY@tok@nf}{\def\PY@tc##1{\textcolor[rgb]{0.00,0.00,1.00}{##1}}}
\@namedef{PY@tok@nc}{\let\PY@bf=\textbf\def\PY@tc##1{\textcolor[rgb]{0.00,0.00,1.00}{##1}}}
\@namedef{PY@tok@nn}{\let\PY@bf=\textbf\def\PY@tc##1{\textcolor[rgb]{0.00,0.00,1.00}{##1}}}
\@namedef{PY@tok@ne}{\let\PY@bf=\textbf\def\PY@tc##1{\textcolor[rgb]{0.80,0.25,0.22}{##1}}}
\@namedef{PY@tok@nv}{\def\PY@tc##1{\textcolor[rgb]{0.10,0.09,0.49}{##1}}}
\@namedef{PY@tok@no}{\def\PY@tc##1{\textcolor[rgb]{0.53,0.00,0.00}{##1}}}
\@namedef{PY@tok@nl}{\def\PY@tc##1{\textcolor[rgb]{0.46,0.46,0.00}{##1}}}
\@namedef{PY@tok@ni}{\let\PY@bf=\textbf\def\PY@tc##1{\textcolor[rgb]{0.44,0.44,0.44}{##1}}}
\@namedef{PY@tok@na}{\def\PY@tc##1{\textcolor[rgb]{0.41,0.47,0.13}{##1}}}
\@namedef{PY@tok@nt}{\let\PY@bf=\textbf\def\PY@tc##1{\textcolor[rgb]{0.00,0.50,0.00}{##1}}}
\@namedef{PY@tok@nd}{\def\PY@tc##1{\textcolor[rgb]{0.67,0.13,1.00}{##1}}}
\@namedef{PY@tok@s}{\def\PY@tc##1{\textcolor[rgb]{0.73,0.13,0.13}{##1}}}
\@namedef{PY@tok@sd}{\let\PY@it=\textit\def\PY@tc##1{\textcolor[rgb]{0.73,0.13,0.13}{##1}}}
\@namedef{PY@tok@si}{\let\PY@bf=\textbf\def\PY@tc##1{\textcolor[rgb]{0.64,0.35,0.47}{##1}}}
\@namedef{PY@tok@se}{\let\PY@bf=\textbf\def\PY@tc##1{\textcolor[rgb]{0.67,0.36,0.12}{##1}}}
\@namedef{PY@tok@sr}{\def\PY@tc##1{\textcolor[rgb]{0.64,0.35,0.47}{##1}}}
\@namedef{PY@tok@ss}{\def\PY@tc##1{\textcolor[rgb]{0.10,0.09,0.49}{##1}}}
\@namedef{PY@tok@sx}{\def\PY@tc##1{\textcolor[rgb]{0.00,0.50,0.00}{##1}}}
\@namedef{PY@tok@m}{\def\PY@tc##1{\textcolor[rgb]{0.40,0.40,0.40}{##1}}}
\@namedef{PY@tok@gh}{\let\PY@bf=\textbf\def\PY@tc##1{\textcolor[rgb]{0.00,0.00,0.50}{##1}}}
\@namedef{PY@tok@gu}{\let\PY@bf=\textbf\def\PY@tc##1{\textcolor[rgb]{0.50,0.00,0.50}{##1}}}
\@namedef{PY@tok@gd}{\def\PY@tc##1{\textcolor[rgb]{0.63,0.00,0.00}{##1}}}
\@namedef{PY@tok@gi}{\def\PY@tc##1{\textcolor[rgb]{0.00,0.52,0.00}{##1}}}
\@namedef{PY@tok@gr}{\def\PY@tc##1{\textcolor[rgb]{0.89,0.00,0.00}{##1}}}
\@namedef{PY@tok@ge}{\let\PY@it=\textit}
\@namedef{PY@tok@gs}{\let\PY@bf=\textbf}
\@namedef{PY@tok@ges}{\let\PY@bf=\textbf\let\PY@it=\textit}
\@namedef{PY@tok@gp}{\let\PY@bf=\textbf\def\PY@tc##1{\textcolor[rgb]{0.00,0.00,0.50}{##1}}}
\@namedef{PY@tok@go}{\def\PY@tc##1{\textcolor[rgb]{0.44,0.44,0.44}{##1}}}
\@namedef{PY@tok@gt}{\def\PY@tc##1{\textcolor[rgb]{0.00,0.27,0.87}{##1}}}
\@namedef{PY@tok@err}{\def\PY@bc##1{{\setlength{\fboxsep}{\string -\fboxrule}\fcolorbox[rgb]{1.00,0.00,0.00}{1,1,1}{\strut ##1}}}}
\@namedef{PY@tok@kc}{\let\PY@bf=\textbf\def\PY@tc##1{\textcolor[rgb]{0.00,0.50,0.00}{##1}}}
\@namedef{PY@tok@kd}{\let\PY@bf=\textbf\def\PY@tc##1{\textcolor[rgb]{0.00,0.50,0.00}{##1}}}
\@namedef{PY@tok@kn}{\let\PY@bf=\textbf\def\PY@tc##1{\textcolor[rgb]{0.00,0.50,0.00}{##1}}}
\@namedef{PY@tok@kr}{\let\PY@bf=\textbf\def\PY@tc##1{\textcolor[rgb]{0.00,0.50,0.00}{##1}}}
\@namedef{PY@tok@bp}{\def\PY@tc##1{\textcolor[rgb]{0.00,0.50,0.00}{##1}}}
\@namedef{PY@tok@fm}{\def\PY@tc##1{\textcolor[rgb]{0.00,0.00,1.00}{##1}}}
\@namedef{PY@tok@vc}{\def\PY@tc##1{\textcolor[rgb]{0.10,0.09,0.49}{##1}}}
\@namedef{PY@tok@vg}{\def\PY@tc##1{\textcolor[rgb]{0.10,0.09,0.49}{##1}}}
\@namedef{PY@tok@vi}{\def\PY@tc##1{\textcolor[rgb]{0.10,0.09,0.49}{##1}}}
\@namedef{PY@tok@vm}{\def\PY@tc##1{\textcolor[rgb]{0.10,0.09,0.49}{##1}}}
\@namedef{PY@tok@sa}{\def\PY@tc##1{\textcolor[rgb]{0.73,0.13,0.13}{##1}}}
\@namedef{PY@tok@sb}{\def\PY@tc##1{\textcolor[rgb]{0.73,0.13,0.13}{##1}}}
\@namedef{PY@tok@sc}{\def\PY@tc##1{\textcolor[rgb]{0.73,0.13,0.13}{##1}}}
\@namedef{PY@tok@dl}{\def\PY@tc##1{\textcolor[rgb]{0.73,0.13,0.13}{##1}}}
\@namedef{PY@tok@s2}{\def\PY@tc##1{\textcolor[rgb]{0.73,0.13,0.13}{##1}}}
\@namedef{PY@tok@sh}{\def\PY@tc##1{\textcolor[rgb]{0.73,0.13,0.13}{##1}}}
\@namedef{PY@tok@s1}{\def\PY@tc##1{\textcolor[rgb]{0.73,0.13,0.13}{##1}}}
\@namedef{PY@tok@mb}{\def\PY@tc##1{\textcolor[rgb]{0.40,0.40,0.40}{##1}}}
\@namedef{PY@tok@mf}{\def\PY@tc##1{\textcolor[rgb]{0.40,0.40,0.40}{##1}}}
\@namedef{PY@tok@mh}{\def\PY@tc##1{\textcolor[rgb]{0.40,0.40,0.40}{##1}}}
\@namedef{PY@tok@mi}{\def\PY@tc##1{\textcolor[rgb]{0.40,0.40,0.40}{##1}}}
\@namedef{PY@tok@il}{\def\PY@tc##1{\textcolor[rgb]{0.40,0.40,0.40}{##1}}}
\@namedef{PY@tok@mo}{\def\PY@tc##1{\textcolor[rgb]{0.40,0.40,0.40}{##1}}}
\@namedef{PY@tok@ch}{\let\PY@it=\textit\def\PY@tc##1{\textcolor[rgb]{0.24,0.48,0.48}{##1}}}
\@namedef{PY@tok@cm}{\let\PY@it=\textit\def\PY@tc##1{\textcolor[rgb]{0.24,0.48,0.48}{##1}}}
\@namedef{PY@tok@cpf}{\let\PY@it=\textit\def\PY@tc##1{\textcolor[rgb]{0.24,0.48,0.48}{##1}}}
\@namedef{PY@tok@c1}{\let\PY@it=\textit\def\PY@tc##1{\textcolor[rgb]{0.24,0.48,0.48}{##1}}}
\@namedef{PY@tok@cs}{\let\PY@it=\textit\def\PY@tc##1{\textcolor[rgb]{0.24,0.48,0.48}{##1}}}

\def\PYZbs{\char`\\}
\def\PYZus{\char`\_}
\def\PYZob{\char`\{}
\def\PYZcb{\char`\}}
\def\PYZca{\char`\^}
\def\PYZam{\char`\&}
\def\PYZlt{\char`\<}
\def\PYZgt{\char`\>}
\def\PYZsh{\char`\#}
\def\PYZpc{\char`\%}
\def\PYZdl{\char`\$}
\def\PYZhy{\char`\-}
\def\PYZsq{\char`\'}
\def\PYZdq{\char`\"}
\def\PYZti{\char`\~}
% for compatibility with earlier versions
\def\PYZat{@}
\def\PYZlb{[}
\def\PYZrb{]}
\makeatother


    % For linebreaks inside Verbatim environment from package fancyvrb.
    \makeatletter
        \newbox\Wrappedcontinuationbox
        \newbox\Wrappedvisiblespacebox
        \newcommand*\Wrappedvisiblespace {\textcolor{red}{\textvisiblespace}}
        \newcommand*\Wrappedcontinuationsymbol {\textcolor{red}{\llap{\tiny$\m@th\hookrightarrow$}}}
        \newcommand*\Wrappedcontinuationindent {3ex }
        \newcommand*\Wrappedafterbreak {\kern\Wrappedcontinuationindent\copy\Wrappedcontinuationbox}
        % Take advantage of the already applied Pygments mark-up to insert
        % potential linebreaks for TeX processing.
        %        {, <, #, %, $, ' and ": go to next line.
        %        _, }, ^, &, >, - and ~: stay at end of broken line.
        % Use of \textquotesingle for straight quote.
        \newcommand*\Wrappedbreaksatspecials {%
            \def\PYGZus{\discretionary{\char`\_}{\Wrappedafterbreak}{\char`\_}}%
            \def\PYGZob{\discretionary{}{\Wrappedafterbreak\char`\{}{\char`\{}}%
            \def\PYGZcb{\discretionary{\char`\}}{\Wrappedafterbreak}{\char`\}}}%
            \def\PYGZca{\discretionary{\char`\^}{\Wrappedafterbreak}{\char`\^}}%
            \def\PYGZam{\discretionary{\char`\&}{\Wrappedafterbreak}{\char`\&}}%
            \def\PYGZlt{\discretionary{}{\Wrappedafterbreak\char`\<}{\char`\<}}%
            \def\PYGZgt{\discretionary{\char`\>}{\Wrappedafterbreak}{\char`\>}}%
            \def\PYGZsh{\discretionary{}{\Wrappedafterbreak\char`\#}{\char`\#}}%
            \def\PYGZpc{\discretionary{}{\Wrappedafterbreak\char`\%}{\char`\%}}%
            \def\PYGZdl{\discretionary{}{\Wrappedafterbreak\char`\$}{\char`\$}}%
            \def\PYGZhy{\discretionary{\char`\-}{\Wrappedafterbreak}{\char`\-}}%
            \def\PYGZsq{\discretionary{}{\Wrappedafterbreak\textquotesingle}{\textquotesingle}}%
            \def\PYGZdq{\discretionary{}{\Wrappedafterbreak\char`\"}{\char`\"}}%
            \def\PYGZti{\discretionary{\char`\~}{\Wrappedafterbreak}{\char`\~}}%
        }
        % Some characters . , ; ? ! / are not pygmentized.
        % This macro makes them "active" and they will insert potential linebreaks
        \newcommand*\Wrappedbreaksatpunct {%
            \lccode`\~`\.\lowercase{\def~}{\discretionary{\hbox{\char`\.}}{\Wrappedafterbreak}{\hbox{\char`\.}}}%
            \lccode`\~`\,\lowercase{\def~}{\discretionary{\hbox{\char`\,}}{\Wrappedafterbreak}{\hbox{\char`\,}}}%
            \lccode`\~`\;\lowercase{\def~}{\discretionary{\hbox{\char`\;}}{\Wrappedafterbreak}{\hbox{\char`\;}}}%
            \lccode`\~`\:\lowercase{\def~}{\discretionary{\hbox{\char`\:}}{\Wrappedafterbreak}{\hbox{\char`\:}}}%
            \lccode`\~`\?\lowercase{\def~}{\discretionary{\hbox{\char`\?}}{\Wrappedafterbreak}{\hbox{\char`\?}}}%
            \lccode`\~`\!\lowercase{\def~}{\discretionary{\hbox{\char`\!}}{\Wrappedafterbreak}{\hbox{\char`\!}}}%
            \lccode`\~`\/\lowercase{\def~}{\discretionary{\hbox{\char`\/}}{\Wrappedafterbreak}{\hbox{\char`\/}}}%
            \catcode`\.\active
            \catcode`\,\active
            \catcode`\;\active
            \catcode`\:\active
            \catcode`\?\active
            \catcode`\!\active
            \catcode`\/\active
            \lccode`\~`\~
        }
    \makeatother

    \let\OriginalVerbatim=\Verbatim
    \makeatletter
    \renewcommand{\Verbatim}[1][1]{%
        %\parskip\z@skip
        \sbox\Wrappedcontinuationbox {\Wrappedcontinuationsymbol}%
        \sbox\Wrappedvisiblespacebox {\FV@SetupFont\Wrappedvisiblespace}%
        \def\FancyVerbFormatLine ##1{\hsize\linewidth
            \vtop{\raggedright\hyphenpenalty\z@\exhyphenpenalty\z@
                \doublehyphendemerits\z@\finalhyphendemerits\z@
                \strut ##1\strut}%
        }%
        % If the linebreak is at a space, the latter will be displayed as visible
        % space at end of first line, and a continuation symbol starts next line.
        % Stretch/shrink are however usually zero for typewriter font.
        \def\FV@Space {%
            \nobreak\hskip\z@ plus\fontdimen3\font minus\fontdimen4\font
            \discretionary{\copy\Wrappedvisiblespacebox}{\Wrappedafterbreak}
            {\kern\fontdimen2\font}%
        }%

        % Allow breaks at special characters using \PYG... macros.
        \Wrappedbreaksatspecials
        % Breaks at punctuation characters . , ; ? ! and / need catcode=\active
        \OriginalVerbatim[#1,codes*=\Wrappedbreaksatpunct]%
    }
    \makeatother

    % Exact colors from NB
    \definecolor{incolor}{HTML}{303F9F}
    \definecolor{outcolor}{HTML}{D84315}
    \definecolor{cellborder}{HTML}{CFCFCF}
    \definecolor{cellbackground}{HTML}{F7F7F7}

    % prompt
    \makeatletter
    \newcommand{\boxspacing}{\kern\kvtcb@left@rule\kern\kvtcb@boxsep}
    \makeatother
    \newcommand{\prompt}[4]{
        {\ttfamily\llap{{\color{#2}[#3]:\hspace{3pt}#4}}\vspace{-\baselineskip}}
    }
    

    
    % Prevent overflowing lines due to hard-to-break entities
    \sloppy
    % Setup hyperref package
    \hypersetup{
      breaklinks=true,  % so long urls are correctly broken across lines
      colorlinks=true,
      urlcolor=urlcolor,
      linkcolor=linkcolor,
      citecolor=citecolor,
      }
    % Slightly bigger margins than the latex defaults
    
    \geometry{verbose,tmargin=1in,bmargin=1in,lmargin=1in,rmargin=1in}
    
    

\begin{document}
    
    \maketitle
    
    

    
    

    \section{Homework 3 (Due 07 Feb)}\label{homework-3-due-07-feb}

\textbf{Due Feb 7 (midnight)}

Total points: \textbf{100}.

    \subsection{Introduction to homework
3}\label{introduction-to-homework-3}

This week's sets of classical pen and paper and computational exercises
deal with the motion of different objects under the influence of various
forces. The relevant reading background is 1. chapter 2 of Taylor (there
are many good examples there) and

\begin{enumerate}
\def\labelenumi{\arabic{enumi}.}
\setcounter{enumi}{1}
\tightlist
\item
  chapters 5-7 of Malthe-Sørenssen.
\end{enumerate}

In both textbooks there are many nice worked out examples.
Malthe-Sørenssen's text contains also several coding examples you may
find useful.

There are several pedagogical aims we have in mind with these exercises:
1. Get practice in setting up and analyzing a physical problem, finding
the forces and the relevant equations to solve;

\begin{enumerate}
\def\labelenumi{\arabic{enumi}.}
\setcounter{enumi}{1}
\item
  Analyze the results and ask yourself whether they make sense or not;
\item
  Finding analytical solutions to problems if possible and compare these
  with numerical results. This teaches us also how to understand errors
  in numerical calculations;
\item
  Being able to solve (in mechanics these are the most common types of
  equations) numerically ordinary differential equations and compare the
  solutions where possible with analytical solutions;
\item
  Getting used to studying physical problems using all possible tools,
  from paper and pencil to numerical solutions;
\item
  Then analyze the results and ask yourself whether they make sense or
  not.
\end{enumerate}

The above steps outline important elements of our understanding of the
scientific method. Furthermore, there are also explicit coding skills we
aim at such as setting up arrays, solving differential equations
numerically and plotting your results. Coding practice is also an
important aspect. The more we practice the better we get (hopefully).
From a numerical mathematics point of view, we will solve the
differential equations using Euler's method (forward Euler).

The code we will develop can be reused as a basis for coming homeworks.
We can also extend the numerical solver we write here to include other
methods (later) like the modified Euler method (Euler-Cromer, midpoint
Euler) and more advanced methods like the family of Runge-Kutta methods
and the Velocity-Verlet method.

At the end of this course, we will thus have developed a larger code (or
set of codes) which will allow us to study different numerical methods
(integration and differential equations) as well as being able to study
different physical systems. Combined with analytical skills, the hope is
that this can allow us to explore interesting and realistic physics
problems. By doing so, the hope is that can lead to deeper insights
about the laws of motion which govern a system.

And hopefully you can reuse many of the above solvers in other courses
(our ideal).

\subsubsection{Practicalities about homeworks and
projects}\label{practicalities-about-homeworks-and-projects}

\begin{enumerate}
\def\labelenumi{\arabic{enumi}.}
\item
  You can work in groups (optimal groups are often 2-3 people) or by
  yourself. If you work as a group you can hand in one answer only if
  you wish. \textbf{Remember to write your name(s)}!
\item
  Homeworks are available ten days before the deadline.
\item
  How do I(we) hand in? You can hand in the paper and pencil exercises
  as a \textbf{single scanned PDF document}. For this homework this
  applies to exercises 1-5. Your jupyter notebook file should be
  converted to a \textbf{PDF} file, attached to the same PDF file as for
  the pencil and paper exercises. All files should be uploaded to
  Gradescope.
\end{enumerate}

\textbf{\href{../resources/gradescope-submissions.md}{Instructions for
submitting to Gradescope}.}

    \subsection{Exercise 1 (20 pt), Electron moving into an electric
field}\label{exercise-1-20-pt-electron-moving-into-an-electric-field}

An electron is sent through a varying electrical field. Initially, the
electron is moving in the \(x\)-direction with a velocity \(v_x = 100\)
m/s. The electron enters the field when it passes the origin. The field
varies with time, causing an acceleration of the electron that varies in
time

    \[
\vec{a}(t)=\left(-20\mathrm{m/s}^2-10\mathrm{m/s}^3t\right) \vec{e}_y
\]

    \begin{itemize}
\item
  1a (4pt) Find the velocity as a function of time for the electron.
\item
  1b (4pt) Find the position as a function of time for the electron.
\end{itemize}

The field is only acting inside a box of length \(L = 2m\). * 1c (4pt)
For how long time is the electron inside the field?

\begin{itemize}
\item
  1d (4pt) What is the displacement in the \(y\)-direction when the
  electron leaves the box. (We call this the deflection of the
  electron).
\item
  1e (4pt) Find the angle the velocity vector forms with the horizontal
  axis as the electron leaves the box.
\end{itemize}

    \subsection{Exercise 2 (10 pt), Drag
force}\label{exercise-2-10-pt-drag-force}

We can observe that the models for linear and quadratic drag forces are
given by:

\[f_{lin} = 3\pi \eta D v \qquad f_{quad} = \kappa \rho A v^2\]

where \(D\) is the ``length scale'' of the object (e.g., the diameter of
the sphere), \(\eta\) is the viscosity of the fluid, \(\rho\) is the
density of the fluid, \(A\) is the cross-sectional area of the object,
and \(\kappa\) is a constant of order unity (larger for flat and blunt
bodies; smaller for streamlined bodies).

\begin{itemize}
\tightlist
\item
  2a (5pt) The Reynolds number is defined as \(Re = \rho v D / \eta\).
  What is the physical meaning of this number? For a spherical object,
  show that the ratio of the quadratic drag force to the linear drag
  force is given by \(f_{quad}/f_{lin} = Re/48\). Use this to explain
  the physical meaning of the Reynolds number. \textbf{Note: you may
  assume that \(\kappa = 0.25\) for a sphere.}
\item
  2b (2pt) Explain a situation where there would be a low Reynolds
  number. What about a high Reynolds number? Estimate the Reynolds
  number for a falling rain drop, a parachutist, a car, and a plane.
\item
  2c (3pt) Find another
  \href{https://en.wikipedia.org/wiki/Dimensionless_numbers_in_fluid_mechanics}{dimensionless
  number from fluid mechanics} and explain its physical meaning. Make
  sure you consider high and low values of the number you find.
\end{itemize}

    \subsection{Exercise 3 (10 pt), Falling
object}\label{exercise-3-10-pt-falling-object}

We have shown that an object that is dropped from rest and experiences
linear drag force will reach a terminal velocity and do so
exponentially:

\[v_y(t) = v_{ter}(1-e^{-t/\tau}).\]

\begin{itemize}
\tightlist
\item
  3a (4pt) At first, the object will be moving slowly. Show that we can
  approximate this expression in that interval. We should find that
  \(v_y = gt\) -- the standard result for free fall in vacuum.
  Demonstrate this.
\item
  3b (3pt) What is the next term, \(O(t^2)\), in the expansion? What is
  the physical meaning of this term?
\item
  3c (4pt) The position of the object is given by:
  \(y(t) = v_{ter}t + (v_{y0}-v_{ter})\tau(1-e^{-t/\tau})\). Show that
  this reduces to the standard result \(y = \frac{1}{2}gt^2\) when
  \(v_{y0} = 0\). What is the small parameter in your expansions in all
  cases?
\end{itemize}

    \subsection{Exercise 4 (10 pt), and now a theoretical
exercise}\label{exercise-4-10-pt-and-now-a-theoretical-exercise}

Finding and exploring equations of motion is the central enterprise of
classical mechanics. You will encounter (and derive) many equations you
have not seen before, and you will need to explain them. Consider a
hypothetical one-dimensional system where a mass \(m\) has a speed of
\(v_0\) and coasts along the x-axis. The surrounding medium produces a
drag force that is modeled using:

\[F(v) = -c v^{3/2}.\]

When the force is a pure function of velocity, using the technique
called separation of variables on Newton's 2nd Law, we can find the
velocity as a function of time:

We can separate variables and write:

\[F(v) = m a = m \dfrac{dv}{dt}\]

Divide both sides by \(F(v)\) and multiply both sides by \(dt\):

\[dt = \dfrac{m}{F(v)}dv\]

And then, given starting (initial) conditions, we integrate both sides
to find an expression for \(t\) as a function of \(v\).

\[t = m \int_{v_0}^v \dfrac{dv'}{F(v')}\]

\begin{itemize}
\tightlist
\item
  4a (4pt) For the given force above, write the equation of motion in a
  form: \(\dfrac{m dv}{F(v)} = dt\). Integrate both sides to find an
  expression for the velocity \(v\) as a function of \(t\) (\(v(t)\) not
  \(t(v)\)).
\item
  4b (2pt) Check your answer by looking at the limits of its behavior.
  Does it agree with your intuition?
\item
  4c (2pt) Sketch the expression and explain what the motion of the
  object looks like by interpreting this expression and your sketch.
\item
  4d (2pt) Will the object ever come to rest?
\end{itemize}

    \subsection{Exercise 5 (10 pt), back to a falling ball and preparing for
the numerical
exercise}\label{exercise-5-10-pt-back-to-a-falling-ball-and-preparing-for-the-numerical-exercise}

\textbf{Useful material: Malthe-Sørenssen chapter 7.5 and Taylor chapter
2.4.}

In this example we study the motion of an object subject to a constant
force and a velocity dependent force. We will reuse the code we develop
here in homework 4 for a position-dependent force.

Here we limit ourselves to a ball that is thrown from a height \(h\)
above the ground with an initial velocity \(\vec{v}_0\) at time
\(t=t_0\). We assume the air resistance is proportional to the square
velocity. Together with the gravitational force these are the forces
acting on our system.

\texttt{\{note\}\ Due\ to\ the\ specific\ velocity\ dependence,\ we\ cannot\ find\ an\ analytical\ solution\ for\ motion\ in\ the\ \$x\$\ and\ \$y\$\ directions,\ see\ the\ discussion\ in\ Taylor\ after\ eq.\ (2.61).}

In order to find an analytical solution we need to assume that the
object is falling in the \(y\)-direction (negative direction) only.

The position of the ball as function of time is \(\vec{r}(t)\) where
\(t\) is time. The position is measured with respect to a coordinate
system with origin at the floor.

We assume we have an initial position \(\vec{r}(t_0)=h\vec{e}_y\) and an
initial velocity \(\vec{v}_0=v_{x,0}\vec{e}_x+v_{y,0}\vec{e}_y\).

In this exercise we assume the system is influenced by the gravitational
force

    \[
\vec{F}_{grav}=-mg\vec{e}_y
\]

    and an air resistance given by a square law

    \[
\vec{F}_{drag} = -Dv\vec{v}.
\]

    The analytical expressions for velocity and position as functions of
time will be used to compare with the numerical results in exercise 6.

\begin{itemize}
\item
  5a (3pt) Identify the forces acting on the ball and set up a diagram
  with the forces acting on the ball. Find the equation of motion for
  the falling ball. \textbf{Do not limit to 1D yet!}
\item
  5b (4pt) Assume now that the object is falling only in the
  \(y\)-direction (negative direction). Integrate the equation of motion
  from an initial time \(t_0\) to a final time \(t\) and find the
  velocity. Assume it is dropped from rest to simplify the mathematics.
  In Taylor equations (2.52) to (2.58) you will find a very good
  discussion of this.
\item
  5c (3pt) Find thereafter the position as function of time starting
  with an initial time \(t_0\). Find the time it takes to hit the floor.
  Here you will find it convenient to set the initial velocity in the
  \(y\)-direction to zero. Taylor equations (2.52)-(2.58) should contain
  all relevant information for solving this part as well.
\end{itemize}

We will use the above analytical results in our numerical calculations
in exercise 6. The analytical solution in the \(y\)-direction only will
serve as a test for our numerical solution. \textbf{We don't often know
the solutions to our problems exactly, so we have to check them against
things we do know.}

    \subsection{Exercise 6 (40pt), Numerical elements, solving exercise 5
numerically}\label{exercise-6-40pt-numerical-elements-solving-exercise-5-numerically}

\textbf{This exercise should be handed in as a jupyter-notebook} at D2L.
Remember to write your name(s).

Last week we: 1. Gained more practice with plotting in Python

\begin{enumerate}
\def\labelenumi{\arabic{enumi}.}
\setcounter{enumi}{1}
\tightlist
\item
  Became familiar with arrays and representing vectors with such objects
\end{enumerate}

This week we will: 1. Learn and utilize Euler's Method to find the
position and the velocity

\begin{enumerate}
\def\labelenumi{\arabic{enumi}.}
\setcounter{enumi}{1}
\item
  Compare analytical and computational solutions
\item
  Add additional forces to our model
\end{enumerate}

    \begin{tcolorbox}[breakable, size=fbox, boxrule=1pt, pad at break*=1mm,colback=cellbackground, colframe=cellborder]
\prompt{In}{incolor}{1}{\boxspacing}
\begin{Verbatim}[commandchars=\\\{\}]
\PY{o}{\PYZpc{}}\PY{n}{matplotlib} \PY{n}{inline}

\PY{c+c1}{\PYZsh{} let\PYZsq{}s start by importing useful packages we are familiar with}
\PY{k+kn}{import}\PY{+w}{ }\PY{n+nn}{numpy}\PY{+w}{ }\PY{k}{as}\PY{+w}{ }\PY{n+nn}{np}
\PY{k+kn}{import}\PY{+w}{ }\PY{n+nn}{matplotlib}\PY{n+nn}{.}\PY{n+nn}{pyplot}\PY{+w}{ }\PY{k}{as}\PY{+w}{ }\PY{n+nn}{plt}
\PY{o}{\PYZpc{}}\PY{n}{matplotlib} \PY{n}{inline}
\end{Verbatim}
\end{tcolorbox}

    We will choose the following values 1. mass \(m=0.2\) kg

\begin{enumerate}
\def\labelenumi{\arabic{enumi}.}
\setcounter{enumi}{1}
\item
  accelleration (gravity) \(g=9.81\) m/s\(^{2}\).
\item
  initial position is the height \(h=2\) m
\item
  initial velocities \(v_{x,0}=v_{y,0}=10\) m/s
\end{enumerate}

Can you find a reasonable value for the drag coefficient \(D\)? You need
also to define an initial time and the step size \(\Delta t\). We can
define the step size \(\Delta t\) as the difference between any two
neighboring values in time (time steps) that we analyze within some
range. It can be determined by dividing the interval we are analyzing,
which in our case is time \(t_{\mathrm{final}}-t_0\), by the number of
steps we are taking \((N)\). This gives us a step size
\(\Delta t = \dfrac{t_{\mathrm{final}}-t_0}{N}\).

With these preliminaries we are now ready to plot our results from
exercise 5.

\begin{itemize}
\tightlist
\item
  6a (10pt) Set up arrays for time, velocity, acceleration and positions
  for the results from exercise 5. Define an initial and final time.
  Choose the final time to be the time when the ball hits the ground for
  the first time. Make a plot of the position and velocity as functions
  of time. Here you could set the initial velocity in the
  \(y\)-direction to zero and use the result from exercise 5. Else you
  need to try different initial times using the result from exercise 5
  as a starting guess. It is not critical if you don't reach the ground
  when the initial velocity in the \(y\)-direction is not zero.
\end{itemize}

We move now to the numerical solution of the differential equations as
discussed in the
\href{https://mhjensen.github.io/Physics321/doc/pub/motion/html/motion.html}{lecture
notes} or Malthe-Sørenssen chapter 7.5. Let us remind ourselves about
Euler's Method.

Suppose we know \(f(t)\) and its derivative \(f'(t)\). To find
\(f(t+\Delta t)\) at the next step, \(t+\Delta t\), we can consider the
Taylor expansion:

\(f(t+\Delta t) = f(t) + \dfrac{(\Delta t)f'(t)}{1!} + \dfrac{(\Delta t)^2f''(t)}{2!} + ...\)

If we ignore the \(f''\) term and higher derivatives, we obtain

\(f(t+\Delta t) \approx f(t) + (\Delta t)f'(t)\).

This approximation is the basis of Euler's method, and the Taylor
expansion suggests that it will have errors of \(O(\Delta t^2)\). Thus,
one would expect it to work better, the smaller the step size \(h\) that
you use. In our case the step size is \(\Delta t\).

In setting up our code we need to

\begin{enumerate}
\def\labelenumi{\arabic{enumi}.}
\item
  Define and obtain all initial values, constants, and time to be
  analyzed with step sizes as done above (you can use the same values)
\item
  Calculate the velocity using \(v_{i+1} = v_{i} + (\Delta t)*a_{i}\)
\item
  Calculate the position using \(pos_{i+1} = r_{i} + (\Delta t)*v_{i}\)
\item
  Calculate the new acceleration \(a_{i+1}\).
\item
  Repeat steps 2-4 for all time steps within a loop.
\end{enumerate}

\begin{itemize}
\item
  6b (20 pt) Write a code which implements Euler's method and compute
  numerically and plot the position and velocity as functions of time
  for various values of \(\Delta t\). Comment your results.
\item
  6c (10pt) Compare your numerically obtained positions and velocities
  with the analytical results from exercise 5. In order to do this, you
  need to take out the motion in the \(x\)-direction. Comment again your
  results.
\end{itemize}

    \subsubsection{Integrating Classwork With
Research}\label{integrating-classwork-with-research}

This opportunity will allow you to earn up to 5 extra credit points on a
Homework per week. These points can push you above 100\% or help make up
for missed exercises. In order to earn all points you must:

\begin{enumerate}
\def\labelenumi{\arabic{enumi}.}
\item
  Attend an MSU research talk (recommended research oriented Clubs is
  provided below)
\item
  Summarize the talk using at least 150 words
\item
  Turn in the summary along with your Homework (Email to
  \href{mailto:caball14@msu.edu}{\nolinkurl{caball14@msu.edu}}).
\end{enumerate}

Approved talks: Talks given by researchers through the following clubs:

\begin{itemize}
\item
  Society for Physics Students (SPS)\hspace{0pt}: Meets Monday Nights
  (alternates with Astronomy Club)
\item
  Astronomy Club\hspace{0pt}: Meets Monday Nights (alternates with SPS)
\item
  Any \href{https://pa.msu.edu/news-events-seminars/index.aspx}{physics
  and astronomy seminar} of interest to you
\end{itemize}

If you have any questions please consult Danny.

    


    % Add a bibliography block to the postdoc
    
    
    
\end{document}
